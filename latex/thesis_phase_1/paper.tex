\documentclass[12pt,a4paper]{report}
% \usepackage{import}
\usepackage{templates/mainpreambel}

\newcommand{\attention}[1]{\color{red}\textbf{[#1]}\color{black}\unskip}
\newcommand{\OR}[1]{\color{red}\textbf{[OR]}\color{black}\unskip}
\newcommand{\optional}[1]{\color{blue}\textbf{~#1}\color{black}\unskip}

\newcommand{\needsciteempty}{CITE}
\newcommand{\needscitecomment}[1]{#1}
\newcommand{\needscite}[1]{\color{purple}\textbf{\textsuperscript{
    CITE #1\ignorespacesafterend
}}\color{black}}
\newcommand{\cProbCurrState}{\cprob{z_t}{t, z_{1:T}, u_{t}, \theta_h}}
\newcommand{\cProbNextState}{\cprob{z_{t+1}}{t, z_{1:T}, u_{1:T}, x_{1:T}, \theta_h}}
\newcommand{\cProbCurrObservation}{\cprob{x_t}{t, z_{1:T}, u_{1:T}, \theta_f}}

\newcommand{\cProbCurrShortState}{\cprob{z_t}{z_{1:t}, u_{t}, \theta_h}}
\newcommand{\cProbNextShortState}{\cprob{z_{t+1}}{z_{1:t}, u_{1:t}, \theta_h}}
\newcommand{\cProbCurrShortObservation}{\cprob{x_t}{z_{1:t}, \theta_f}}



% https://en.wikibooks.org/wiki/LaTeX/Glossary
% https://www.overleaf.com/learn/latex/Glossaries
% https://tex.stackexchange.com/questions/199211/differences-between-xindy-and-makeindex
% https://tex.stackexchange.com/a/541990
% https://tools.ietf.org/doc/texlive-doc/latex/glossaries/glossariesbegin.html
\makeglossaries

\DeclareLanguageMapping{american}{american-apa}
% \addbibresource{./references/bibliography.bib}
\addbibresource{./references/autoupdated.bib}
\loadglsentries[acronym]{./references/glossary.tex}
\graphicspath{{figures/}}

\usepackage{subfiles} 
                                                                                         




\begin{document}


%%% Title page
\subfile{content/cover.tex}
\subfile{content/abstract.tex}
% \import{templates/}{abstract.tex}


\tableofcontents
\printglossary[type=acronym, title=List of terms, toctitle=List of terms]

% TODO: Apply title case to all chapters, sections and subsections.
% TODO: Change 'will' to an abbreviated version 

% Status -> Issues -> existing approach -> limitations of approach -> challenge -> RQ 
\chapter{Introduction}
\label{sec:intro}

\section{Motivation}
\label{sec:motivation}
\subfile{content/sections/sec_motivation}

\section{Problem Space}
\label{sec:challenges}
\subfile{content/sections/sec_challenges}

\section{Outline}
\label{sec:approaches}
\subfile{content/sections/sec_approaches}

% \import{content/sections/}{sec_issues_01.tex}

\chapter{Background}
\label{ch:prereq}
This chapter will explore the most important concepts for this work. Most of the concepts can have several meanings depending on the varying context in which they are applied. For this purpose, we will provide an intuitive understanding, the ensuing challenges, a concrete definition for this work and lastly and a mathematically formal description. The concepts we will cover encompase \optional{sequence modelling}, process mining and counterfactual explanations.


\section{Process Mining}
\label{sec:process}
\subfile{content/sections/sec_pm}

\section{Multivariate Time-Series Modelling}
\label{sec:sequences}
\subfile{content/sections/sec_mlts}


\section{Counterfactuals}
\label{sec:counterfactuals}
\subfile{content/sections/sec_cf}


\section{Related Literature}
\label{sec:literature}
\subfile{content/sections/sec_literature}


\section{Research Question}
\label{sec:rq}
\subfile{content/sections/sec_rq}

\section{Formal Definitions}
\label{sec:formulas}
\subfile{content/sections/sec_formula}







\chapter{Methods}
\label{sec:methods}

\section{Validity Measure}
\label{sec:validity}
\subfile{content/sections/sec_validity}

\section{Models}
\label{sec:models}
\subfile{content/sections/sec_models}

\section{Datasets}
\label{sec:datasets}
\subfile{content/sections/sec_datasets}

\chapter{Results}
\label{sec:results}

\section{Evaluation Procedure}
\label{sec:evaluation}

\subsection{}

\chapter{Discussion}
\label{sec:dicussion}


\chapter{Conclusion}
\label{sec:conclusion}

% \glsaddall % Just to add all glossary entries, for exemplary purposes

\printbibliography

\appendix

\end{document}

