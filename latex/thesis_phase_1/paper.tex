\documentclass[12pt,a4paper]{report}
\usepackage{import}
\usepackage{templates/mainpreambel}
\newcommand{\attention}[1]{\color{red}\textbf{[#1]}\color{black}\unskip}
\newcommand{\OR}[1]{\color{red}\textbf{[OR]}\color{black}\unskip}
\newcommand{\optional}[1]{\color{blue}\textbf{~#1}\color{black}\unskip}

\newcommand{\needsciteempty}{CITE}
\newcommand{\needscitecomment}[1]{#1}
\newcommand{\needscite}[1]{\color{purple}\textbf{\textsuperscript{
    CITE #1\ignorespacesafterend
}}\color{black}}

% https://en.wikibooks.org/wiki/LaTeX/Glossary
% https://www.overleaf.com/learn/latex/Glossaries
% https://tex.stackexchange.com/questions/199211/differences-between-xindy-and-makeindex
% https://tex.stackexchange.com/a/541990
% https://tools.ietf.org/doc/texlive-doc/latex/glossaries/glossariesbegin.html
\makeglossaries

\DeclareLanguageMapping{american}{american-apa}
% \addbibresource{./references/bibliography.bib}
\addbibresource{./references/autoupdated.bib}
\loadglsentries[acronym]{./references/glossary.tex}

\begin{document}


%%% Title page
\import{templates/}{cover.tex}
\import{templates/}{abstract.tex}


\tableofcontents
\printglossary[type=acronym, title=List of terms, toctitle=List of terms]

% TODO: Apply title case to all chapters, sections and subsections.
% TODO: Change 'will' to an abbreviated version 

% Status -> Issues -> existing approach -> limitations of approach -> challenge -> RQ 
\chapter{Introduction}
\label{sec:intro}

\import{content/sections/}{sec_context_01.tex}

\import{content/sections/}{sec_issues_01.tex}

\chapter{Background}
\label{sec:prereq}
This chapter will explore the most important concepts for this work. Most of the concepts can have several meanings depending on the varying context in which they are applied. For this purpose, we will provide an intuitive understanding, the ensuing challenges, a concrete definition for this work and lastly and a mathematically formal description. The concepts we will cover encompase \optional{sequence modelling}, process mining and counterfactual explanations.


\section{Process Mining}
\label{sec:process}

\subsection{A definition for Business Processes}
Before elaborating on Process Mining, we have to establsih the meaning of the term \emph{process} in the context of this paper. The term is broadly used in many contexts and therefore has a rich semantic volume. A process generally refers to something that advances and changes over time\autocite{_DefinitionPROCESS_}.
Although, legal or biological processes may be valid understandings, we focus on processes \emph{business processes}.

An example is a loan application process in which an applicant may request a loan at a specific point in time. The case is then assessed and reviewed by multiple approvers and ends in a final decision. The loan may be granted or denied. The \emph{business} part may be misleading as these processes are not confined to commercial settings. For instance, a medical business process may cover a patients admission to a hospital, followed by a series of diagnostics and treatments and ending with the recovery or death of a patient. Another example from a human-computer-interaction\attention{Add to glossary} perspective would be an order process for an online retail service like Amazon. The buyer might start the process by adding articles to the shopping cart and proceeding with specifying their bank account details. This order process would end with the submission or receival of the order.

All of these examples have a number of common characteristics. They have a clear starting point which is followed by numerous intermediary steps and end in one of the possible sets of outcomes.For this paper we will refer to each step, including start and end points, as \gls{event}. Each \gls{event} may contain additional information in the form of event attributes. A collection of these \glspl{event} refer to a \gls{instance}, if they all relate to a single run of a process. In line with the aforementioned examples, these \glspl{instance} could be understood as a single loan application, a medical case or a buy order. We can also attach \gls{instance} related information to each instance. Examples would be the applicants race, a patients age or the buyers budget. In its entirety, a business process can be summarised as a \emph{graph} or \emph{flowchart}, in which every node represents an event and each arc the path to another event. This graphical representation is referred to as \emph{process map}. \autoref{fig:example_process} shows an example of such a representation.


\begin{figure}[htb]
    \centering
    \includegraphics[width=0.25\textwidth]{figures/misc/placeholder.png}
    \caption{This graph shows an example of various process maps.}
    \label{fig:example_process}
\end{figure}



\noindent In conclusion, in this thesis a \emph{business process} refers to \begin{quote}
    \emph{A finite series of discrete events with one or more starting points, intermediary steps and end points.}
\end{quote}
However, we have to address a number of issues with this definition.
First, this definition excludes infinite processes like \attention{XXX} or continuous processes such as \attention{XXX}. There may be valid arguments to include processes with these characteristics, but they are not relevant for this thesis.
Second, in each example we deliberately used words that accentuate modality such as \emph{may}, \emph{can} or \emph{would}. It is important to understand that each process anchors its definition in its application context. Hence, what defines a business process is indisputably subjective. For instance, while an online marketplace like Amazon might be interested in the process from the customers first click to the successful shipment, an Amazon vendor might be interested in the delivery process of a product only.
Third, the example provided in \autoref{fig:example_process} may not relate to the reality of a data generating process. In line with the second point, these examples subjective models of a process. They may or may not be accurate. The \emph{true} process is often unknown to every actor. Therefore, we will distinguish between the \emph{true process model} and a \emph{process model}. The \emph{true process model} is a hypothetical concept whose \emph{true} structure remains unknown.

\subsection{What is Process Mining}
Having established our understanding of a process, we can turn towards \emph{Process Mining}. This young discipline has many connections to other fields that focus on the modeling and analysis of processes such as \gls{CPI} or \gls{BPM}. However, its data-centric approaches originate in \gls{dm}.
The authors \citeauthor{vanderaalst_ProcessMiningManifesto_2012} describe this field as a discipline \enquote{to discover, monitor and improve real processes (i.e., not assumed processes) by extracting knowledge from event logs readily available in today's (information) systems}\autocite{vanderaalst_ProcessMiningManifesto_2012}. The discipline revolves around the analysis of \glspl{log}.
% TODO: Minor mistake with information systems
An \gls{log} is a collection of \glspl{instance}. These logs are retrievable from various sources like an \glspl{IS} or database. Those logs are often stored in data formats such as \gls{CSV} or \gls{XES}.

\subsection{The Challenges of Process Mining}
As mentioned in \autoref{sec:intro}, process data modelling and analysis is a challenging task. \citeauthor{vanderaalst_ProcessMiningManifesto_2012} mentions a number of issues that arise from processes\autocite{vanderaalst_ProcessMiningManifesto_2012}.

The first issue arises from the quality of the data set. Process logs are seldomly collected with the primary goal of mining information and hence, often appear to be of subpar quality. The information is often in complete due to a lack of context information, the ommision of logged process steps or wrong levels of granularity.

This issue is exacerbated by the second major issue with process data. Mainly, its complexity. Not only does a process logs complexity arise from the variety of data sources and differing levels of complexity, but also from the datas' characteristics. The data can often be viewed as multivariate sequence with discrete and continuous features and variable length. This characteristic alone creates problems explored in \autoref{sec:sequences}. \attention{Also refer to variability in sequence section.} However, the data is also just a \emph{sample} of the process. Hence, it may not reflect the real process in its entirety. In fact, mining techniques need to incorporate the \emph{open world assumption} as the original process may generate unseen \glspl{instance}.

A third issue which contributes to the datasets incompleteness and complexity is a phenomenon called \emph{concept drift}. This phenomenon relates possibility of a change in the \emph{true} process. The change may occur suddenly or gradually and can appear in isolation or periodically. An expression of such a drift may be a sudden inclusion of a new process step or domain changes of certain features. These changes are not uncommon and their likelihood increases with the temporal coverage and level of granularity of the dataset\needscite. In other words, the more \emph{time} the dataset covers and the higher its detail, the more likely a change might have occured over the time.

All three issues relate to the \emph{representativeness} of the data with regards to the unknown \emph{true} process that generated the data. However, they also represent open challenges that require research on their own. For our purpose, we have to assume that the data is representative and its underlying process is static. These assumptions are widely applied in the body of process mining literature\needscite.


\section{Multivariate Time-Series Modelling}
\label{sec:sequences}
The data which is mined in Process Mining is typically a multivariate time-series. It is important to establish the characteristics of time-series.

\subsection{What are Time Series Models?}
A time series can be understood as a series of observable values, that depend on previous values. The causal dependence turns time-series into a special case of sequence models. Sequences do not \emph{have to} depend on previous values. They might depend on previous and future values or not be interdependent at all. An example of a sequence model would be a language model. Results in \gls{NLP}, that the words in a sentences for many languages do not seem to only depend on prior words but also on future words\needscite. Hence, we can assume that a human has formulated his sentence in the brain before expressing it in a sequence of words\needscite. In contrast to sequences, time series cannot depend on future values. The general understanding of \emph{time} is linear and forward directed\needscite. The notion of time relates to our understanding of \emph{cause and effect}. Hence, we can decompose any time series in a precedent (causal) and an antecedent (effect) part\needscite. A time series model attempts to capture the relationship between precedent and antecedent. 

\subsection{The Challenges of Time Series Modelling}
The analysis of unrestricted sequential opens up a myriad of challenges. First, sequential data introduce a combinatorial set of possible realisations. For instance, a set of two objects $\{A,B\}$ yields 7 theoretical combinations ($\{\emptyset\}$, $\{A\}$, $\{B\}$, $\{A,B\}$, $\{B,A\}$, $\{A,A\}$, $\{B,B\}$). Just by adding C and D to the object set increases the number of combinations to 40 and then 341. % ABCD => 341 
Second, sequential data may contain cycles which increases the number of possible productions to infinity\needscite. Both, the combinatorial increase and cycles, contain a set of a countable infinite number of possbilities for discrete sets. However, as processes may also contain additional information a third obstacle arises. Including additional information increases the set to an uncountable number of possible values. With these obstacles in mind, it often becomes intractable to compute an exact model. 

Hence, we have to include restrictive assumptions to reduce the solution space to a tractable number. A common way to counter this combinatorial explosion is the inclusion of the \emph{Granger Causality} assumption. This idea postulates the predictive capability of a sequence given its preceding sequence. In other words, if we know that D can only follow after C, then 341 possible combinations reduce to 170. All of these possible 170 combinations are now temporally-related and hence, we speak of a \emph{time-series}. 

However, the prediction of sequences raises two new questions. First, if we know the precedence of a time-series, what is the antecedent? And second, if we can predict the antecedent accurately, what caused it?
At first glance, it is easy to believe that both questions are quite similar, because we could assume that the precedent causes the antecedent. However, 
the first question is often solved using predictive AI models that rely on data, like Hidden-Markov-Models or Deep Learning. However, the latter question is much more difficult as data cannot help explain causes of sequences that never occured. To illustrate this impossibility, if we never encountered 'D' in our dataset consisting of A, B and C, we cannot reliably say what would cause D. Answering this question requires additional tools within the \gls{xai} framework. One such method is the focus of this thesis and is further explored in \autoref{sec:counterfactuals}.   

\subsection{A Process within the State-Space Framework}
% https://en.wikipedia.org/wiki/State-space_representation
Generally speaking, every time-series can be represented as a state-space model \needscite. Within this framework the system consists of \emph{input states} for \emph{subsequent states} and \emph{subsequent outputs}. The general mathematical form of such a system is shown in \autoref{eq:nonlinear_state_space}. 

\begin{equation}
    \label{eq:nonlinear_state_space}
    \begin{array}{l}
        \mathbf{x'}(t)=\mathbf{f}(t, x(t), u(t)) \\
        \mathbf{y}(t)=\mathbf{h}(t, x(t), u(t))
    \end{array}
\end{equation}

Here, u represents the input, x the state, t the time. The function f maps t, x(t) and u(t) to a another state x'. y acts as an output computed by function h which takes the same input as f. The distinction x' and y is used to decouple the state, which may be \emph{hidden}, from \emph{observable} system outputs. \autoref{fig:nonlinear_state_space} shows a graphical representation of these equations. 
\needsfigure{fig:nonlinear_state_space}{This figure shows a simplified graphical representation of a state-space model. Each arrow represents the flow of information.}

The body of literature for state-space models is too vast to discuss them in detail\footnote{For an introduction to state-space models see: XXX}. However, for process mining we can use this representation to discuss the necessary assumptions with regards to process mining. 

In accordance to the definition in \autoref{sec:process}, we can understand the \gls{log} as a collection of observable outputs of a state-space model. The state of the process is hidden as the \emph{true} process which generated the data cannot be observed as well. The time t is a step within the process and u is represents inputs of the process. Here, u subsumes observable information about the process. The functions f and h control the transitions of a process' state to another state and its output (\gls{instance}). As we establish in \autoref{sec:process}, we can assume that a process is a discrete sequence, whose transitions are time-variant. These assumtions simplify the representation to a linear system of equations as depicted in \autoref{eq:linear_state_space}  

\begin{equation}
    \label{eq:linear_state_space}
    \begin{array}{l}
    \mathbf{x}(k+1)=\mathbf{A}(k) \mathbf{x}(k)+\mathbf{B}(k) \mathbf{u}(k) \\
    \mathbf{y}(k)=\mathbf{C}(k) \mathbf{x}(k)+\mathbf{D}(k) \mathbf{u}(k)
    \end{array}
\end{equation}

This representation extends \autoref{eq:nonlinear_state_space} by including time dependent transition matrices A, B, C and D. We refer to this model as \emph{explicit discrete time-variant} system. The transition matrices are time-dependant, as a process' transition may change at every time step. Note that k replaces t to express the \emph{discreteness} of each time step. Also note the change of x'(t) to x(k+1) as it introduces a causal dependence on the previous state x(k). \attention{A number of AI techniques where developed to model this representation bla bla bla (HMM, Kalman, etc).} \attention{ALSO MENTION additional assumptions or reductions.}

\section{Counterfactuals}
\label{sec:counterfactuals}
Counterfactuals are an important explanatory tool to understand a models' cause for decisions. Generating counter factuals is main focus of this thesis. Hence, we will establish the most important chateristics of counterfactuals in this section.

\subsection{What are Counterfactuals?}
Counterfactuals have various definitions. However, their semantic meaning refers to \enquote{a conditional whose antecedent is false}\autocite{_Counterfactual_}. A simpler definition from \citeauthor{starr_Counterfactuals_2021} states, counterfactual modality concerns itself with \emph{what is not, but could or would have been}.
Both definitions are related to linguistics and philosophy. Within AI and the mathematical framework various formal definitions can be found within causal inference\autocite{hitchcock_CausalModels_2020}. Here, citeauthor describes a counterfactual as {Causal inference definition}. What binds all of these definitions is the notion of causality within "what if" scenarios. 

However, for this paper, we will use the understanding established within the \gls{xai} context. Within \gls{xai}, counterfactuals act as a prediction which \enquote{describes the smallest change to the feature values that changes the prediction to a predefined output}\autocite{molnar2019}. Note that \gls{xai} mainly concerns the explanation of models, which are always subject to inductive biases of the model itself and therefore inherently subjective. The idea behind counterfactuals as explanations\footnotemark ~is that we understand the output of a model, if we know what change caused would cause a different outcome.\footnotetext{There are other explanatory techniques in XAI like \emph{feature importances} but counterfactuals are considered the most human-understandable} For instance, lets denote a sequence 1 as \textit{ABCDE\textbf{FG}}, then a counterfactual \textit{ABCDE\textbf{XZ}} would tell us that \textbf{F} (probably) caused \textbf{G} in sequence 1. As counterfactuals only address explanations of single model instances and not the model as a whole, they are called \emph{local} explanation. 

\emph{Valid} counterfactuals satisfy four criteria. First, a counterfactual should be minimally different from the true instance. If the counterfactual to sequence 1 was \textit{A\textbf{A}CDE\textbf{XZ}} we would already have difficulties to discern whether B or F or both caused G at the end of sequence 1. Second, a counterfactual should produce a predefined outcome as closely as possible. This characteristic is ingrained in \citeauthor{molnar2019}s definition. If the counterfactual \textit{ABCDE\textbf{XZ}} ends with Z but this sequence is highly unrealistic, then cannot be certain of our conclusion for sequence 1. Third, we typically desire multiple diverse counterfactuals.  One counterfactual might not be enough to understand the causal relationships in a sequence. In the example above we might have a clue that F causes G but what if G is not only caused by F? If we are able to find counterfactuals \textit{\textbf{V}BCDEF\textbf{H}} and \textit{ABCDE\textbf{XZ}} but all other configurations lead to G, then we know positions 1 and 6 cause G. As last criterion, each counterfactual should be possible. A sequence \textit{ABCDE\textbf{1G}} would not be possible if numericals are not allowed. All four criteria allow us to assess the validity of each generated counterfactual and thus, help us to define an evaluation metric. 

\subsection{The Challenges of Counterfactual Sequence Generation}
The current literature surounding counterfactuals expose a number of challenges when dealing with counterfactuals. 

The most important disadvantage of counterfactuals is the \gls{rashomon}\autocite[ch. 9.3]{molnar2019}. If all of the counterfactuals are valid, but contradict eachother, we have to decide which of the \emph{truths} are worth considering. 

This decision reveals the next challenge of evaluation\needscite. Although, the criteria can support us with the decision, it remains a question \emph{how} to evaluate counterfactuals. Every automated measure comes with implicit assumptions and often do not guarantee a realistic explanation. We still need domain experts to assess their validity. 

The generation of counterfactual sequences contribute to both former challenges, due to the combinatorial expansion of the solution space. This problem is common for counterfactual sentence generation and has been adressed within the \gls{NLP}\needscite. However, as process mining data not only consist of discrete objects like \emph{words}, but also event and case features, the problem remains a daunting task. So far, little work has gone into the generation of multivariate counterfactual sequences like \glspl{instance}\needscite.

% Note that the \gls{XAI} definition, differs from the 

% \attention{They all share the question of "what if", which is always highly subjective with regards to the assumptions made. This will seep into the remainder of the paper.}


\section{Related Literature}
\optional{Rationality - Counterfactual thinking play a crucial role in planning actions}

\section{Formal Definitions}
To formalise  the log and a process model, we will use the formalisation established by \citeauthor{???}\needscite.

\section{Research Question}

\section{General Approach}

\section{What is Process Mining?}

\section{Challenges of Processing Process Data}

\chapter{Related Papers}


\chapter{Methods}
\label{sec:methods}

\section{Datasets}
\label{sec:datasets}

\section{Preprocessing}
\label{sec:preprocessing}

\section{Framework}

\chapter{Results}
\label{sec:results}

\section{Evaluation}
\label{subsec:evaluation}


\chapter{Discussion}
\label{sec:dicussion}


\chapter{Conclusion}
\label{sec:conclusion}

% \glsaddall % Just to add all glossary entries, for exemplary purposes

\printbibliography

\appendix

\end{document}

