\documentclass[12pt,a4paper]{report}
\usepackage{import}
\usepackage{templates/mainpreambel}
\newcommand{\attention}[1]{\color{red}\textbf{[#1]}\color{black}\unskip}
\newcommand{\OR}[1]{\color{red}\textbf{[OR]}\color{black}\unskip}
\newcommand{\optional}[1]{\color{blue}\textbf{~#1}\color{black}\unskip}

\newcommand{\needsciteempty}{CITE}
\newcommand{\needscitecomment}[1]{#1}
\newcommand{\needscite}[1]{\color{purple}\textbf{\textsuperscript{
    CITE #1\ignorespacesafterend
}}\color{black}}

% https://en.wikibooks.org/wiki/LaTeX/Glossary
% https://www.overleaf.com/learn/latex/Glossaries
% https://tex.stackexchange.com/questions/199211/differences-between-xindy-and-makeindex
% https://tex.stackexchange.com/a/541990
% https://tools.ietf.org/doc/texlive-doc/latex/glossaries/glossariesbegin.html
\makeglossaries

\DeclareLanguageMapping{american}{american-apa}
% \addbibresource{./references/bibliography.bib}
\addbibresource{./references/autoupdated.bib}
\loadglsentries[acronym]{./references/glossary.tex}

\begin{document}


%%% Title page
\import{templates/}{cover.tex}
\import{templates/}{abstract.tex}


\tableofcontents
\printglossary[type=acronym, title=List of terms, toctitle=List of terms]

% TODO: Apply title case to all chapters, sections and subsections.
% TODO: Change 'will' to an abbreviated version 

% Status -> Issues -> existing approach -> limitations of approach -> challenge -> RQ 
\chapter{Introduction}
\label{sec:intro}

\import{content/sections/}{sec_context_01.tex}

\import{content/sections/}{sec_issues_01.tex}

\chapter{Background}
\label{sec:prereq}
This chapter will explore the most important concepts for this work. Most of the concepts can have several meanings depending on the varying context in which they are applied. For this purpose, we will provide an intuitive understanding, the ensuing challenges, a concrete definition for this work and lastly and a mathematically formal description. The concepts we will cover encompase \optional{sequence modelling}, process mining and counterfactual explanations.


\section{What is Process Mining?}
Process Mining is a relatively young discipline with many connections to other fields that focus on the modeling and analysis of processes such as \gls{CPI} or \gls{BPM}. However, its data-centric approaches mainly relate to \gls{dm}.  
\citeauthor{vanderaalst_ProcessMiningManifesto_2012} describes this field as discipline to \enquote{to discover, monitor and improve real processes (i.e., not assumed processes) by extracting knowledge from event logs readily available in today's (information) systems}\autocite{vanderaalst_ProcessMiningManifesto_2012}. The discipline revolves around the analysis of \glspl{log}. 
% TODO: Minor mistake with information systems
An \gls{log} is a collection of \glspl{instance}, which compile related \glspl{event}. These logs are retrievable from various sources like an \glspl{IS} or database. Those logs are often stored in data formats such as \gls{CSV} or \gls{XES}.  

\subsection{The difficulties of Process Mining}


\subsection{Formal Definition}
To formalise  the log and a process model, we will use the formalisation established by \citeauthor{???}\needscite.

\section{What are Counterfactuals?}
% Aim: Status -> Issues ->   

Counterfactuals have various definitions. However, their semantic meaning refers to \enquote{a conditional whose antecedent is false}\autocite{_Counterfactual_}. A simpler definition from \citeauthor{starr_Counterfactuals_2021} states, counterfactual modality concerns itself with \emph{what is not, but could or would have been}.
Both definitions are related to linguistics and philosophy. Within AI and the mathematical framework various formal definitions can be found within causal inference\autocite{hitchcock_CausalModels_2020}. However, for this paper, we will use the understanding established within the \acrfull{xai} context\footnote{\attention{XAI is a discipline which seeks to develop techniques to better understand machine learning models.}}. Within \acrshort{xai}, counterfactuals act as a prediction which \enquote{describes the smallest change to the feature values that changes the prediction to a predefined output}\autocite{molnar2019}. 
\subsection{Related Literature}
\optional{Rationality - Counterfactual thinking play a crucial role in planning actions}

\subsection{Definition in the context of this Thesis}
\subsection{Formal Definition}


\attention{Causal inference definition}, \attention{XAI definition}.
One can understand this as prediction of "what" happens "if" a precursing event would have been different.
\attention{They all share the question of "what if", which is always highly subjective with regards to the assumptions made. This will seep into the remainder of the paper.}

\attention{What are counterfactuals?}

Counterfactuals are commonly to relate to questions about the outcomes of situations that

Hence, we want to minimally edit a process to understand the changes necessary to achieve an alternative outcome.



\section{Research Question}

\section{General Approach}

\section{What is Process Mining?}

\section{Challenges of Processing Process Data}

\chapter{Related Papers}


\chapter{Methods}
\label{sec:methods}

\section{Datasets}
\label{sec:datasets}

\section{Preprocessing}
\label{sec:preprocessing}

\section{Framework}

\chapter{Results}
\label{sec:results}

\section{Evaluation}
\label{subsec:evaluation}


\chapter{Discussion}
\label{sec:dicussion}


\chapter{Conclusion}
\label{sec:conclusion}

% \glsaddall % Just to add all glossary entries, for exemplary purposes

\printbibliography

\appendix

\end{document}

