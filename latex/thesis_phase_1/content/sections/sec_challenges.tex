\documentclass[./../../paper.tex]{subfiles}
\graphicspath{{\subfix{./../../figures/}}}

\begin{document}

In this paper, we will approach the problem of generating counterfactuals for processes. The literature has provided a multitude of techniques to generate counterfactuals for AI models, that are derived from static data\footnote{With static data, we refer to data that does not change over a time dimension.}. However, little research has focussed on counterfactuals for dynamic data\footnote{With dynamic data, we refer to data that has time as a major component, which is also inherently sequential}. A major reason, emerges from a multitude of challenges, when dealing with counterfactuals and sequences. 

Three of these challenges are particularly important. First, counterfactuals within AI attempt to explain outcomes which never occured. A \emph{what-if} questions often refer to hypothetical scenarios. Therefore, there is no evidential data, from which we can infer predictions. Subsequently, this lack of evidence further complicates the evaluation of generated counterfactuals. In other words, you cannot validate the correctness of a theoretical outcome that has never occured.
Second, sequential data is highly variable in length, but processes steps have complicated factors, too\needscite{}. The sequential nature of the data impedes the tractability of many problems due to the combinatorial explosion of possible sequences. Furthermore, the data generated is seldomly one-dimensional or discrete. Henceforth, each dimension's contribution can vary in dependance of its context, the time and magnitude.
Third, process data often requires knowledge of the causal structures that produced the data in the first place. However, these structures are often hidden and it is a NP-hard problem to elicit them\needscite{Check process discovery literature}. 
Just these challenges, makes the field in which we can contribute a vast endeavor. 

\end{document}
