\section{Context of this Thesis}
Many processes, often medical, economical, or administrative in nature, are governed by sequential events and their contextual environment. Many of these events and their order of appearance play a crucial part in the determination of every possible outcome. With the rise of AI and the increased abundance of data in recent years several techqniques emerged that help to predict the outcomes of complex processes in the real world.\attention{Expand the domain application.}
For instance, research in the Process Mining discipline has shown that is possible to predict the outcome of a particular process fairly well\needscite{}. \attention{However, while the models created to predict certain outcomes, in many cases, we also aim to change that outcome into a different one.}  \attention{We also aim to reduce the effort with which we want to bring forth the desired outcome}. In other words, we want to change the outcome of a particular event, by making it maximally likely, with as little interventions as possible. In this thesis we will investigate a way to achieve this goal with the use of counterfactuals. Counterfactuals within the AI framework help us to answer theoretical "what-if" questions. Knowing these answers will help us further understand what to do to avoid or enforce the outcome of a process.\attention{WHY}