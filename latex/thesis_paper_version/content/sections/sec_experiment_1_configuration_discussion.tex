\documentclass[./../../paper.tex]{subfiles}
\graphicspath{{\subfix{./../../figures/}}}

\begin{document}


The results show us that the initiation procedure heavily determines the starting point of the algorithm. Hence, this result is hardly surprising. We have discussed the reasons in \autoref{sec:model_evolutionary}. Namely, more sophisticated methods than random initiation can heavily influence the starting point of the evolutionary algorithm and determine how fast the algorithm reaches convergence. 

Interestingly, among the top-5 configurations, only the 5th operation has the \emph{Elitism-Selector} and the \emph{Fittest-Survivor-Recombiner}. Both operators heavily focus on deterministic selections of the very best individuals. The fact that only one of these approaches reached the top tells us that this combination is naturally prone to local maxima. We can also see how much faster it ran and converged at its highest viability. Therefore, \emph{CBI-RWS-OPC-SBM-BBR} is much more interesting as it did not converge after 100 iterative cycles. Hence, it is likely it may reach higher viability scores if we choose to let it run longer.

It is equally interesting that the best model turned out to be the one that sorts the individuals based on a given sorting order. As we chose the order in favour of the least impactful viability component (feasibility), this may suggest that ranked sorting may act as a suitable balancing mechanism. 

The monotonous increase of the feasibility may have two possible reasons. Either, this behaviour displays the bias within the feasibility component. We mentioned that the feasibility is biased towards shorter sequences. Hence, the feasibility might increase until only one event is left. Therefore, there might not even be a convergence. Another reason could be that the more dominant viability components are optimised first and, afterwards, the feasibility. Hence, after 100 iterations, there is still much room to improve. In other words, we would have to increase the termination point before encountering convergence. The results with regards to the recombiners provide a clue. Here, we see that the \emph{Fittest-Survivor-Recombiner} and \emph{Best-Breed-Recombiner} do converge on feasibility, while the {Ranked-Recombiner} does not. In other words, we lose a lot of potential because the algorithm prioritises other components. 

% \subsection{Discussion}
% The reasons for the superiority of \attention{FactualInitiator} are clear. Suppose we start the model with the factuals as the initial population. In that case, the factual will already have a viability of at least two as similarity and sparsity have to be at their maximal value. As the prediction model tends to only assign scores close to the extremes, the favourable change of an event attribute often yields a strong bias which is often correct. Hence, the viabilities often reach viability of around 3. The only way to reach higher viability for factually inititiated counterfactuals is to approach the Pareto-surface by increasing the feasibility. In other words, one would have to increase feasibility without significantly decreasing the similarity, sparsity and improvement scores. Similarly, it is no surprise that the \attention{FactualInitiator} harms the feasibility, as it is difficult to find a case that is even more likely than a case directly sampled from the log.

% Moving forward, we have to choose a set of operators. We consider the following operators: We choose the \optional{Case-Based-Initiator} as it might increase our chances of generating feasible variables.

% For selection, we use the \optional{Elitism-Selector}, as it generally appears to return better results.
% Furthermore, we include the \attention{FactualInitiator}, as it would be interesting whether we can reach better results by changing parameters. For selection, we use the \attention{ElitismSelector and RouletteWheelSelector}. The former because it seems to be consistently better than the other selectors. The latter because we suspect that the negative effect is highly biased by the results of the \attention{FactualInitiator}. 

% Furthermore, we choose to move forward with the \optional{One-Point-Crosser}. This crossing operation is slightly better in yielding feasible results.

% For selection and recombination, we use the \optional{Elitism-Selector} and the \optional{FittestSurvivorRecombiner}, respectively. They all outcompete their alternatives for all similarity, sparsity and feasibility.

% In the next experiment, we vary mutation rates using \ModelEVOFSR.



\end{document}