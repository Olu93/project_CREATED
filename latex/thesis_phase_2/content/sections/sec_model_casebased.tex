\documentclass[./../../paper.tex]{subfiles}
\graphicspath{{\subfix{./../../figures/}}}

\begin{document}

Case-based techniques leverage the data by using example instances. The idea is to find suitable candidates that fulfill the counterfactual criterions the best. We treat this model as a baseline. Therefore we keep this approach simple. We find candidates by searching by randomly sampling cases from the Log and then, evaluating them using the viability measure.

Inherently, this approach is restricted by the \emph{representativeness} of the data. It is not possible to generate counterfactuals that have not been seen before. This method works for cases, in which the data holds enough information about the process. If this condition is not met, it is impossible to produce suitable candidates.

Note, that this approach will automatically fulfill the criterion of being feasible, as the counterfactuals are drawn from the Log directly. Hence, we expect their feasibility to often be higher than for other methods.


\end{document}