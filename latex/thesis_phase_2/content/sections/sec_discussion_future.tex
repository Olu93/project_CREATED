\documentclass[./../../paper.tex]{subfiles}
\graphicspath{{\subfix{./../../figures/}}}

\begin{document}
Regarding future directions, it is worth pointing out whether employing other components of the viability structure is beneficial. The measure described here clearly operationalised a set of criteria. However, there may be more aspects to consider and generate even better counterfactuals. A good example would be diversity. In terms of other evolutionary approaches, applying modern state-of-the-art methods with the same viability measure would be interesting.

% It would be interesting to see the out put of using \citeauthor{hsieh_DiCE4ELInterpretingProcess_2021}'s way to measure viability.
% \subsection{Inclusion of a better Measure for Sequential Feasibility}
% If the measure is probabilistic one could try to compute the expected probability given the sequence instead of just the probability.
% Furthermore, one could measure the feasibility by employing perplexity or another NLP evaluation metric. 

\end{document}