\documentclass[./../../paper.tex]{subfiles}
\graphicspath{{\subfix{./../../figures/}}}

\begin{document}

In this thesis, we use a multitude of datasets for generating the counterfactuals. All of the data sets where taken from \citeauthor{teinemaa_OutcomeOrientedPredictiveProcess_2018a}. Each dataset consists of log data and contains labels which signify the outcome of a process. They were introduced by \attention{some author}. We focus on binary outcome predictions. Hence, each dataset will provide information about one of two possible outcomes associated with the case. For instance, a medical process might be deemed a success if the patient is cured or a failure if the patient remains ill. A loan application process might deem granting the loan a success or the rejection as failure. The determination of the outcome depends on the use-case and the stakeholders involved. A insurence provider might deem a successful claim as a failure, while the client deems it as a success.

\begin{enumerate}
    \item[BPIC12] The first dataset is the popular BPIC12 dataset. This dataset was originally published for the Business Process Intelligence Conference and contains events for a loan application process. Each indivdual case relates to one loan application process and can be accepted (regular) or cancelled (deviant).
    \item[Sepsis] The next dataset is the Sepsis-Dataset. It is a medical dataset, which records of patients with life-threatening sepsis conditions. The outcome describes whether the patient returns to the emergency room within 28 days from initial discharge.
    \item[TrafficFines] Third, we apply our approach to the Traffic-Fines-Dataset. This dataset contains events related to notifications sent related to a fine. The dataset originates in a log from an Italian local police force.
    \item[Dice4EL] Lastly, we include a variation of the BPIC dataset. It is the dataset which was used by \citeauthor{hsieh_DiCE4ELInterpretingProcess_2021}. The difference between this dataset and the original dataset is two-fold. First, \citeauthor{hsieh_DiCE4ELInterpretingProcess_2021} omit most variables except two. Second it is primarily designed for next-activity prediction and not outcome prediction. We modified the dataset, to fit the outcome prediction model.
\end{enumerate}

For more information about these datasets we refer to \citeauthor{teinemaa_OutcomeOrientedPredictiveProcess_2018a}'s comparative study\autocite{teinemaa_OutcomeOrientedPredictiveProcess_2018a}.

Below we list all the important descriptive statistics in \autoref{tbl:datasets}. \attention{num deviant and num regular should be based on the counts within the cases.} \attention{Time should just get seconds not this format.}

% In order to increase the speed of our experiments we limit the maximum sequence length to 25 events. We refer to this dataset as \emph{BPIC12-25}. We also apply subsequent experiments on the same dataset with a maximum sequence length of 50\attention{To show validity regardless of sequence length}. This dataset will be refered to as \emph{BPIC12-50}. 

Lastly, we apply our approach to a third dataset of a different domain\attention{To show validity across datasets}.\attention{Add a dataset description here.} Below we list all the important descriptive statistics in \autoref{tbl:datasets}. \attention{num deviant and num regular should be based on the counts within the cases.} \attention{Time should just get seconds not this format.}

% \begin{widepage}
\begin{tabular}{lrrrrrrrrlrr}
 & Num Cases & Min Seq Len & Max Seq Len & Ratio Distinct Traces & Num Distinct Events & Num Data Columns & Num Event Features & Preprocessing Time & Time Unit & Num Outcome Regular & Num Outcome Deviant \\
Dataset &  &  &  &  &  &  &  &  &  &  &  \\
Dice4EL & 3728 & 15 & 50 & 0.000268 & 36 & 9 & 7 & 0.438000 & seconds & 2111 & 1617 \\
BPIC12-25 & 866 & 15 & 25 & 0.001155 & 32 & 23 & 21 & 0.511004 & seconds & 682 & 184 \\
BPIC12-50 & 3728 & 15 & 50 & 0.000268 & 36 & 25 & 23 & 2.311005 & seconds & 2111 & 1617 \\
BPIC12-75 & 4461 & 15 & 75 & 0.000224 & 36 & 25 & 23 & 2.239135 & seconds & 2379 & 2082 \\
BPIC12-100 & 4628 & 15 & 100 & 0.000216 & 36 & 25 & 23 & 4.326515 & seconds & 2420 & 2208 \\
Sepsis25 & 707 & 5 & 25 & 0.001414 & 15 & 75 & 73 & 1.313337 & seconds & 610 & 97 \\
Sepsis50 & 770 & 5 & 47 & 0.001299 & 15 & 76 & 74 & 1.498461 & seconds & 662 & 108 \\
Sepsis75 & 777 & 5 & 66 & 0.001287 & 15 & 76 & 74 & 1.546414 & seconds & 667 & 110 \\
Sepsis100 & 779 & 5 & 88 & 0.001284 & 15 & 76 & 74 & 1.387316 & seconds & 669 & 110 \\
TrafficFines & 129615 & 2 & 20 & 0.000008 & 10 & 40 & 38 & 8.886944 & seconds & 70602 & 59013 \\
\end{tabular}

% \end{widepage}


\end{document}