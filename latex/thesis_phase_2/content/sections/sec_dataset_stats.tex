\documentclass[./../../paper.tex]{subfiles}
\graphicspath{{\subfix{./../../figures/}}}

\begin{document}

In this thesis, we use a multitude of datasets for generating the counterfactuals. All of the data sets where taken from \citeauthor{teinemaa_OutcomeOrientedPredictiveProcess_2018a}. Each dataset consists of log data and contains labels which signify the outcome of a process. They were introduced by \attention{some author}. We focus on binary outcome predictions. Hence, each dataset will provide information about one of two possible outcomes associated with the case. For instance, a medical process might be deemed a success if the patient is cured or a failure if the patient remains ill. A loan application process might deem granting the loan a success or the rejection as failure. The determination of the outcome depends on the use-case and the stakeholders involved. A insurence provider might deem a successful claim as a failure, while the client deems it as a success.

\begin{enumerate}
    \item[BPIC12] The first dataset is the popular BPIC12 dataset. This dataset was originally published for the Business Process Intelligence Conference and contains events for a loan application process. Each indivdual case relates to one loan application process and can be accepted (regular) or cancelled (deviant).
    \item[Sepsis] The next dataset is the Sepsis-Dataset. It is a medical dataset, which records of patients with life-threatening sepsis conditions. The outcome describes whether the patient returns to the emergency room within 28 days from initial discharge.
    \item[TrafficFines] Third, we apply our approach to the Traffic-Fines-Dataset. This dataset contains events related to notifications sent related to a fine. The dataset originates in a log from an Italian local police force.
    \item[Dice4EL] Lastly, we include a variation of the BPIC dataset. It is the dataset which was used by \citeauthor{hsieh_DiCE4ELInterpretingProcess_2021}. The difference between this dataset and the original dataset is two-fold. First, \citeauthor{hsieh_DiCE4ELInterpretingProcess_2021} omit most variables except two. Second it is primarily designed for next-activity prediction and not outcome prediction. We modified the dataset, to fit the outcome prediction model.
\end{enumerate}

\begin{table}[htbp]
    \label{tbl:dataset-stats}
    \caption{All datasets used within the evaluation. Dice4EL is used for the qualitative evaluation and the remaining are used for quantitative evaluation purposes.}
    \begin{adjustbox}{center}
        \begin{tabular}{lllllllll}
Dataset & Dice4EL & BPIC12-25 & BPIC12-50 & BPIC12-75 & BPIC12-100 & BPIC12-Full & Sepsis & TraficFines \\
Num Cases & 3728 & 866 & 3728 & 4461 & 4628 & 4685 & 782 & 129615 \\
Min Seq Len & 15 & 15 & 15 & 15 & 15 & 15 & 5 & 2 \\
Max Seq Len & 50 & 25 & 50 & 75 & 100 & 175 & 185 & 20 \\
Ratio Distinct Traces & 0.000268 & 0.001155 & 0.000268 & 0.000224 & 0.000216 & 0.000213 & 0.001279 & 0.000008 \\
Num Distinct Events & 36 & 32 & 36 & 36 & 36 & 36 & 15 & 10 \\
Num Data Columns & 9 & 23 & 25 & 25 & 25 & 25 & 76 & 40 \\
Num Event Features & 7 & 21 & 23 & 23 & 23 & 23 & 74 & 38 \\
Preprocessing Time & 0:00:00.268037 & 0:00:00.315076 & 0:00:01.083041 & 0:00:01.451540 & 0:00:01.535157 & 0:00:01.738210 & 0:00:01.341117 & 0:00:06.028582 \\
Time Unit & seconds & seconds & seconds & seconds & seconds & seconds & seconds & seconds \\
Num Outcome Regular & 2111 & 682 & 2111 & 2379 & 2420 & 2442 & 671 & 70602 \\
Num Outcome Deviant & 1617 & 184 & 1617 & 2082 & 2208 & 2243 & 111 & 59013 \\
\end{tabular}

    \end{adjustbox}
\end{table}


\noindent For more information about these datasets we refer to \citeauthor{teinemaa_OutcomeOrientedPredictiveProcess_2018a}'s comparative study\autocite{teinemaa_OutcomeOrientedPredictiveProcess_2018a}. We list all the important descriptive statistics in \autoref{tbl:dataset-stats}.\xixi{How should I cite. Is mentioning the authors name enough?}

% In order to increase the speed of our experiments we limit the maximum sequence length to 25 events. We refer to this dataset as \emph{BPIC12-25}. We also apply subsequent experiments on the same dataset with a maximum sequence length of 50\attention{To show validity regardless of sequence length}. This dataset will be refered to as \emph{BPIC12-50}. 

% Lastly, we apply our approach to a third dataset of a different domain\attention{To show validity across datasets}.\attention{Add a dataset description here.} Below we list all the important descriptive statistics in \autoref{tbl:dataset-stats}.



\end{document}