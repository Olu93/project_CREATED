\documentclass[./../../paper.tex]{subfiles}
\graphicspath{{\subfix{./../../figures/}}}

\begin{document}

\section{Determine the best Generator Algorithm}

\subsection{Experimental Setup}
\label{sec:exp4}
Knowing the combinatory set of configurations \autoref{sec:exp1}, the chosen hyperparameters \autoref{sec:exp2} and the optimal number of iterations from \autoref{sec:exp3}, we compare the evolutionary algorithm with other algorithms. 

In this comparison, we other models mentioned in \autoref{sec:models}. Namely, the \emph{Case-Based Generator} and the \emph{Random Generator}. 

We use the resulting viability and feasibility as dependent variables and each generator type as independent variable.

For the evolutionary configurations we choose the configuration set of \autoref{sec:exp1}, the hyperparameters \autoref{sec:exp2} and the number of evolution cycles from \autoref{sec:exp3}. The remaining procedure follows the established procedure of previous experiments.

\subsection{Results}

\needsfigure{fig:num_models_result_1}{This bar-graph shows the effects of increasing the cycle count.}

\needstbl{tbl:viability_params}{This table shows the results of the mixed linear model using viability as dependent variable.}

\needstbl{tbl:feasibility_params}{This table shows the results of the mixed linear model using feasibility as dependent variable.}

In \autoref{fig:num_models_result_1}, we clearly see that \attention{... TBD}. 

\subsection{Discussion}
These results show that the \attention{some model} is clearly superior to the other models. The reason is probably that \attention{... TBD}. Knowing these results, a couple of questions remain. Namely, does the result remain consisten for longer sequences and does the result remain consistent for other datasets. \optional{Furthermore, how does this procedure compare to other methods in the literature.} The remaining experiments will address these issues. 


\end{document}