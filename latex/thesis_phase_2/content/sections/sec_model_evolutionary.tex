\documentclass[./../../paper.tex]{subfiles}
\graphicspath{{\subfix{./../../figures/}}}

\begin{document}
% This section dives into the generative models that we will explore in this thesis. They cover fundamentally different approaches to the counterfactual generation of process data. We apply the viability metric established in \autoref{sec:viability} to evaluate the performance of each model.


% Here, we attempt to capture the latent state-space of a model and use this space to sample counterfactual candidates.  Last, we explore a technique which does not require to optimise a differentiable objective function. Instead we use the viability measure as a fitness function and maximise the fitness of each counterfactual candidate.

We introduced most of the operators in \autoref{sec:evo}. In this section, describe the operators in detail and select a subset that we want to explore further. \xixi{Are implementation details important?}

\subsubsection{Operators}
We implemented a number of different evolutionary operators. Each one belongs to one of five categories. The categories are inititiation, selection, crossing, mutation and recombination.  

\paragraph{Inititation}
\begin{enumerate}
    \item[DI:] The Default-Initiator generates an initial population entirely randomly. 
    \item[SBI:] The \emph{Sampling-Based-Initiation} generates an initial population using a distribution estimated from the data. 
    \item[CBI:] \emph{Case-Based-Initiation} uses examples of the data as initial population. 
    \item[FI:] \emph{CFactual-Initiation} Uses the factual itself. 
\end{enumerate}

\paragraph{Selection}
\begin{enumerate}
    \item[RWI:] \emph{Roulette-Wheel-Selection} Selects individuals randomly, but proportionate to their fitness score. 
    \item[TS:] \emph{Tournament-Selection} Compares two or more individuals and selects a winner among them. 
    \item[ES:] \emph{Elitism-Selection} selects each individual solely on their fitness 
\end{enumerate}

\paragraph{Crossing}
\begin{enumerate}
    \item[OPC:] \emph{One-Point-Crossing} Chooses on point in the sequence and creates offspring by taking everything from or after that point from another individual. 
    \item[TPC:] \emph{Two-Point-Crossing} Chooses two points in the sequence and creates offspring by taking everything between or outside these points from another individual. 
    \item[UCx:] \emph{Uniform-Crossing} Uniformly selects positions in the sequence to take from another individual. The amount of selected positions is determined by a crossing-rate between 0 and 1.
\end{enumerate}
\paragraph{Mutation}
\begin{enumerate}
    \item[RM:] \emph{Random-Mutation} creates entirely random features for inserts and substitution. 
    \item[SBM:] \emph{Sampling-Based-Mutation} creates sampled features based on data distribution for inserts and substitution. 
\end{enumerate}
\paragraph{Recombination}
\begin{enumerate}
    \item[FSR:] \emph{Fittest-Survivor-Recombination} Determines the survivor among the mutated offsprings and the population. 
    \item[BBR:] \emph{Best-of-Breed-Recombination} Determines better than average survivors among the mutated offsprings and adds them to the population. 
    \item[RR:] \emph{Ranked-Recombination} Determines survivors based on ranking 
    % \item[PR:] \emph{Pareto-Ranked-Recombination} Determines survivors based on the pareto principle.
\end{enumerate}


We use abbreviations to refer to them in figure, tables and so one. For instance, \emph{CBI-RWI-OPC-RM-PR} refers to an evolutionary operator configuration, that samples its initial population from the data, probablistically samples parents based on their fitness, crosses them on one point and so on. For the \emph{Uniform-Crossing} operater we additionally indicate its crossing rate using a number. For instance,  \emph{CBI-RWI-UC3-RM-PR} is a model using the \emph{Uniform-Crossing} with a child receiving roughly 30\% of the genom of one parent and 70\% of another parent.


\subsubsection{Model Selection}

\end{document}