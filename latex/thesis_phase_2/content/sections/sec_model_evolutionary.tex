\documentclass[./../../paper.tex]{subfiles}
\graphicspath{{\subfix{./../../figures/}}}

\begin{document}
% This section dives into the generative models that we explore in this thesis. They cover fundamentally different approaches to the counterfactual generation of process data. We apply the viability metric established in \autoref{sec:viability} to evaluate the performance of each model.


% Here, we attempt to capture the latent state-space of a model and use this space to sample counterfactual candidates.  Last, we explore a technique which does not require to optimise a differentiable objective function. Instead we use the viability measure as a fitness function and maximise the fitness of each counterfactual candidate.



We introduced most of the operator-types in \autoref{sec:evo}.
In this section, we describe the concrete set of operators and select a subset that we want to explore further.

For our purposes, the \emph{gene} of a sequence consists of the sequence of events within a \gls{instance}. Hence, if an offspring inherits one gene of a parent, it inherits the activity associated with the event and its event attributes.

\begin{figure}[htbp]
    \begin{tikzpicture}[>=stealth,thick,baseline]



        \matrix [matrix of math nodes, left delimiter=(,right delimiter=)](m1){
            a   & b    & a    \\
            0.6 & 0.25 & 0.70 \\
            0   & 0    & 1    \\
            1.2 & 4.5  & 2.3  \\
        };
        \node[draw,  blue!80, dashed, thin, inner sep=2mm,fit=(m1-1-1.west) (m1-4-2.east)] (attbox) {};
        \node[above = 2mm of m1](rlbl) {Parent 1};
        \node[below = 2mm of attbox, blue!80](lb1) {\tiny Genes passed on};


        \node[right = 2mm of m1](cr){+};

        \matrix [matrix of math nodes,left delimiter=(,right delimiter=), right = 2mm of cr](m2){
            a   & b    & a    & c    \\
            0.6 & 0.75 & 0.64 & 0.57 \\
            0   & 0    & 1    & 0    \\
            1.2 & 4.5  & 3.3  & 3.0  \\
        };
        \node[draw,  red!80, dashed, thin, inner sep=2mm,fit=(m2-1-3.west) (m2-4-4.east)] (attbox-m2) {};
        \node[above = 2mm of m2](rlbl) {Parent 2};
        \node[below = 2mm of attbox-m2, red!80](lb2) {\tiny Genes passed on};


        \node[right = 2mm of m2](eq){=};

        \matrix [matrix of math nodes,left delimiter=(,right delimiter=), right = 2mm of eq](m3){
            a   & b    & a    & c    \\
            0.6 & 0.25 & 0.64 & 0.57 \\
            0   & 0    & 1    & 0    \\
            1.2 & 4.5  & 3.3  & 3.0  \\
        };

        \node[draw,  blue!80, dashed, thin, inner sep=2mm,fit=(m3-1-1.west) (m3-4-2.east)] (attbox-m3-1) {};
        \node[draw,  red!80, dashed, thin, inner sep=2mm,fit=(m3-1-3.west) (m3-4-4.east)] (attbox-m3-2) {};
        \node[below = 2mm of attbox-m3-1, blue!80](lb2) {\tiny Inherited};
        \node[below = 2mm of attbox-m3-2, red!80](lb2) {\tiny Inherited};

        \node[above = 2mm of m3](rlbl) {Offspring};


        % \draw[->, outer] (m2.east) -- (m3.west);

    \end{tikzpicture}
    \caption{A newly generated offspring inheriting genes in the form of activities and event attributes from both parents.}
    \label{fig:example-inheritance}
\end{figure}

Our goal is to generate candidates by evaluating the sequence based on our viability measure. Our measure acts as the fitness function. The candidates that are deemed fit enough are subsequently selected to reproduce offspring. This process is explained in \autoref{fig:example-inheritance}. The offspring is subject to mutations. Then, we evaluate the new population and repeat the procedure until a termination condition is reached. We can directly optimise the viability measure established in \autoref{sec:viability}.

\newcommand{\cf}{\text{counterfactuals}}
\newcommand{\cfp}{\text{cf-parents}}
\newcommand{\cfo}{\text{cf-offsprings}}
\newcommand{\cfm}{\text{cf-mutants}}
\newcommand{\cfs}{\text{cf-survivors}}


\begin{algorithm}[htb!]
    \caption{The basic structure of an evolutionary algorithm.}
    \begin{algorithmic}
        \Require{factual}
        \Require{configuration}
        \Require{sample-size}
        \Require{population-size}
        \Require{mutation-rate}
        \Require{termination-point}
        \Ensure{The result is the final counterfactual sequences}

        \State $counterfactuals \gets initialize(\text{factual})$
        \While{not $termination$}
        \State $\cfp \gets select(\cf, \text{sample-size})$
        \State $\cfo \gets crossover(\cfp) $
        \State $\cfm \gets mutate(\cfo, \text{mutation-rate})$
        \State $\cfs \gets recombine(\cf, \cfm, \text{population-size})$
        \State $termination \gets determine(\cfs, \text{termination-point})$
        \State $\cf \gets \cfs$
        \EndWhile
    \end{algorithmic}
    \label{alg:my-evolutionary}
\end{algorithm}

\subsubsection{Operators}
We implemented a number of different evolutionary operators. Each one belongs to one of five categories. The categories are inititiation, selection, crossing, mutation and recombination.

\paragraph{Inititation}
\begin{enumerate}
    \item[RI:] The \emph{Random-Initiation} generates an initial population entirely randomly. The activity is just a randomly chosen integer and each event attribute is drawn from a normal-distribution.
    \item[SBI:] The \emph{Sampling-Based-Initiation} generates an initial population by sampling from a data distribution estimated from the data directly.
    \item[CBI:] \emph{Case-Based-Initiation} samples individuals from a subset of the Log (\emph{Case-Based-Initiation}). Those individuals are used to initiate the population.
          % \item[FI:] \emph{Factual-Initiation} Uses the factual itself. 
\end{enumerate}
The initiation proceduce might be the most important operation in terms of computaton time. The reason is that we expect more sophisticated initiation procedures like Sampling-Based-Inititation and Case-Based inititiation to start with a much higher viability and therefore, reach their convergence much sooner.


\paragraph{Selection}
\begin{enumerate}
    \item[RWI:] \emph{Roulette-Wheel-Selection} Selects individuals randomly. However, we compute the fitness of each indivdual in the population and choose a random sample proportionate to their fitness values. Hence, sequences with high fitness values have a higher chance to crossover their genes, while less fit individuals also occasionally get their chance.
    \item[TS:] \emph{Tournament-Selection} compares two or more individuals and selects a winner among them. We choose two competing individuals we randomly sample with replacement. Hence, some individuals have multiple chances to compete. The competing individuals are randomly chosen as winners in proportion to their viability. Hence, if an individual with a viability of 3 is pited against an individual with a viability of 1 then there's a 3:1 chance that the first individual will move on to crossover its genes.
    \item[ES:] \emph{Elitism-Selection} selects each individual solely on their fitness. In other words, only a top-k amount of individuals are selected for the next operation. There is no chance for weaker individuals to succeed. This approach is deterministic and therefore, subject to getting stuck in local minima.
\end{enumerate}

\paragraph{Crossing}
\begin{enumerate}
    \item[UCx:] We can uniformly choose a fraction of genes of one individual (\emph{Parent 1}) and overwrite the respective genes of another individual (\emph{Parent 2}). The result is a new individual. We call that (\emph{Uniform-Crossover}).
          \autoref{fig:crossover_uniform} shows a simple schematic example. By repeating this process towards the opposite direction, we create two new offsprings, which share characteristics of both individuals.
          The amount of inherited genes can be adjusted using a rate-factor. The amount of selected positions is determined by a crossing-rate between 0 and 1. The higher the crossover-rate, the higher the risk of disrupting possible sequences. If we turn to \autoref{fig:crossover_uniform} again, we see how the second child has 2 repeating genes at the end. If a process does not allow the transition from \emph{activity 8} to another \emph{activity 8}, then the entire \gls{instance} becomes infeasible.
    \item[OPC:] \emph{One-Point-Crossing} is an approach that is suitable for sequential data of same lengths. We can choose a point in the sequence and pass on genes of \emph{Parent 1} onto the \emph{Parent 2} from that point onwards and backwards (\emph{One-Point-Crossover}).
          Thus, creating two new offsprings as depicted in \autoref{fig:crossover_onepoint}.
    \item[TPC:] \emph{Two-Point-Crossing} resembles its single-point counterpart. However, this time, we choose two points in the sequence and pass on the overlap and the disjoints to generate two new offsprings. Again, \autoref{fig:crossover_twopoint} describes the procedure visually. Obviously, we can increase the number of crossover points even further. However, this increase comes at the risk of disrupting sequential dependencies.
\end{enumerate}

\begin{figure}[htbp]
    \centering
    \begin{tikzpicture}[
            node distance = 4mm,
            MTRX/.style = {matrix of nodes,
                    nodes={draw, minimum size=5.6mm, anchor=center,
                            inner sep=0pt, outer sep=0pt},
                    column sep=-\pgflinewidth,
                    row sep=2mm},
            CR/.style = {fill=red},  %fill  Color Red
            CY/.style = {fill=yellow}%fill  Color Yellow
        ]
        % left table
        \matrix  (m1) [MTRX,
            row 1/.append style = {nodes={CR}},
            row 2/.append style = {nodes={CY}},
            label=below: Parents
        ]
        {
            0 & 1 & 2 & 3 & 4 & 5 & 6 & 7 & 8 & 9 \\
            5 & 8 & 9 & 4 & 2 & 3 & 5 & 7 & 5 & 8 \\
        };
        % right table
        \matrix  (m2) [MTRX, right=of m1,
        label=below:Children
        ]
        {
        |[CR]| 0 & |[CY]| 8 & |[CR]| 2 & |[CR]| 3 & |[CY]| 4
        & |[CR]| 5 & |[CR]| 6 & |[CR]| 7 & |[CY]| 5 & |[CR]| 9   \\
        |[CY]| 5 & |[CR]| 1 & |[CY]| 9 & |[CY]| 4 & |[CR]| 2
        & |[CY]| 3 & |[CY]| 5 & |[CY]| 7 & |[CR]| 8 & |[CY]| 8   \\
        };
        % \draw[very thick, blue] (m1.north) -- (m1.south);
        \draw[very thick, blue] (m1-1-1.north east) |- ++(90:-1.5cm);
        \draw[very thick, blue] (m1-2-1.south east) |- ++(90:1.5cm);        
        \draw[very thick, blue] (m1-1-2.north east) |- ++(90:-1.5cm);
        \draw[very thick, blue] (m1-2-2.south east) |- ++(90:1.5cm);        
        \draw[very thick, blue] (m1-1-4.north east) |- ++(90:-1.5cm);
        \draw[very thick, blue] (m1-2-4.south east) |- ++(90:1.5cm);        
        \draw[very thick, blue] (m1-1-5.north east) |- ++(90:-1.5cm);
        \draw[very thick, blue] (m1-2-5.south east) |- ++(90:1.5cm);        
        \draw[very thick, blue] (m1-1-8.north east) |- ++(90:-1.5cm);
        \draw[very thick, blue] (m1-2-8.south east) |- ++(90:1.5cm);        
        \draw[very thick, blue] (m1-1-9.north east) |- ++(90:-1.5cm);
        \draw[very thick, blue] (m1-2-9.south east) |- ++(90:1.5cm);        
        \draw[double, -{Implies[]}, semithick] (m1.east) -- (m2.west);
    \end{tikzpicture}
    \caption{A crossing process of uniformly applying characteristics of one sequence to another.}
    \label{fig:crossover_uniform}
\end{figure}
\begin{figure}[htbp]
    \centering
    \begin{tikzpicture}[
            node distance = 4mm,
            MTRX/.style = {matrix of nodes,
                    nodes={draw, minimum size=5.6mm, anchor=center,
                            inner sep=0pt, outer sep=0pt},
                    column sep=-\pgflinewidth,
                    row sep=2mm},
            CR/.style = {fill=red},  %fill  Color Red
            CY/.style = {fill=yellow}%fill  Color Yellow
        ]
        % left table
        \matrix  (m1) [MTRX,
            row 1/.append style = {nodes={CR}},
            row 2/.append style = {nodes={CY}},
            label=below: Parents
        ]
        {
            0 & 1 & 2 & 3 & 4 & 5 & 6 & 7 & 8 & 9 \\
            5 & 8 & 9 & 4 & 2 & 3 & 5 & 7 & 5 & 8 \\
        };
        % right table
        \matrix  (m2) [MTRX, right=of m1,
        label=below:Children
        ]
        {
        |[CR]| 5 & |[CR]| 8 & |[CR]| 7 & |[CR]| 8 & |[CR]| 9
        & |[CR]| 3 & |[CR]| 5 & |[CY]| 7 & |[CY]| 8 & |[CY]| 9   \\
        |[CY]| 0 & |[CY]| 1 & |[CY]| 2 & |[CY]| 3 & |[CY]| 4
        & |[CY]| 5 & |[CY]| 6 & |[CR]| 7 & |[CR]| 5 & |[CR]| 8   \\
        };
        % \draw[very thick, blue] (m1.north) -- (m1.south);
        \draw[very thick, blue] (m1-1-7.north east) |- ++(90:-1.5cm);
        \draw[very thick, blue] (m1-2-7.south east) |- ++(90:1.5cm);        
        \draw[double, -{Implies[]}, semithick] (m1.east) -- (m2.west);
    \end{tikzpicture}
    \caption{A figure showing the process of applying characteristics of one sequence to another using one split point}
    \label{fig:crossover_onepoint}
\end{figure}
\begin{figure}[htbp]
    \centering
    \begin{tikzpicture}[
            node distance = 4mm,
            MTRX/.style = {matrix of nodes,
                    nodes={draw, minimum size=5.6mm, anchor=center,
                            inner sep=0pt, outer sep=0pt},
                    column sep=-\pgflinewidth,
                    row sep=2mm},
            CR/.style = {fill=red},  %fill  Color Red
            CY/.style = {fill=yellow}%fill  Color Yellow
        ]
        % left table
        \matrix  (m1) [MTRX,
            row 1/.append style = {nodes={CR}},
            row 2/.append style = {nodes={CY}},
            label=below: Parents
        ]
        {
            0 & 1 & 2 & 3 & 4 & 5 & 6 & 7 & 8 & 9 \\
            5 & 8 & 9 & 4 & 2 & 3 & 5 & 7 & 5 & 8 \\
        };
        % right table
        \matrix  (m2) [MTRX, right=of m1,
        label=below:Children
        ]
        {
        |[CR]| 0 & |[CR]| 1 & |[CY]| 9 & |[CY]| 4 & |[CY]| 2
        & |[CY]| 3 & |[CY]| 5 & |[CR]| 7 & |[CR]| 8 & |[CR]| 9   \\
        |[CY]| 5 & |[CY]| 8 & |[CR]| 2 & |[CR]| 3 & |[CR]| 4
        & |[CR]| 5 & |[CR]| 6 & |[CY]| 7 & |[CY]| 5 & |[CY]| 8   \\
        };
        % https://tex.stackexchange.com/questions/79166/vertical-lines-in-matrix-using-tikz-package
        \draw[very thick, blue] (m1-1-2.north east) |- ++(90:-1.5cm);
        \draw[very thick, blue] (m1-2-2.south east) |- ++(90:1.5cm);
        \draw[very thick, blue] (m1-1-7.north east) |- ++(90:-1.5cm);
        \draw[very thick, blue] (m1-2-7.south east) |- ++(90:1.5cm);
        \draw[double, -{Implies[]}, semithick] (m1.east) -- (m2.west);
    \end{tikzpicture}
    \caption{The process of applying characteristics of one sequence to another using two split points.}
    \label{fig:crossover_twopoint}
\end{figure}



\paragraph{Mutation}
Before elaborating on the details, we have to briefly discuss four modification types that we can apply to sequences of data. Reminiscent of edit distances, which were introduced earlier in this thesis, we can either insert, delete, change and transposition a gene. These edit-types are the fundamental edits we use to modify sequences. For a visual explanation of each edit-type we refer again to \autoref{fig:dl_example} in \autoref{sec:damerau}.

However, we can change the extent to which each operations is applied over the sequence. We call these parameters \emph{mutation-rates}. In other words, if the delete-rate equals 1 every individual experiences a modification which results in the deletion of a step. Same applies to other edit types.

As we chose the hybrid encoding scheme, we have to define what an insert or a change means for the data. Aside from changing the activity, we also have to choose a new set of data attributes. This necessity requires to define two ways to produce them. We can either choose the features randomly, or choose to take a more sophisticated  approach.

\begin{enumerate}
    \item[RM:] \emph{Random-Mutation} creates entirely random features for inserts and substitution. The activity is just a randomly chosen integer and each event attribute is drawn from a normal-distribution.
    \item[SBM:] \emph{Sampling-Based-Mutation} creates sampled features based on data distribution for inserts and substitution. We can simplify the approach by invoking the \emph{Markov Assumption} and sample the feature attributes given the activity in question (\emph{Sample-Based-Mutation}).

\end{enumerate}
% Here, we apply only one of the two operations. However, the extend in which these mutations are applicable can still vary.

There are still two noteworthy topics to discuss.

First, these edit-types are disputable. One can argue, that change and transpose are just restricted versions of delete-insert compositions. For instance, if we want to change the activity \emph{Buy-Order} with \emph{Postpone-Order} at timestep 4, we can first, delete \emph{Buy-Order} and insert \emph{Postpone-Order} at the same place. Similar holds for transpositions, albeit more complex. Hence, these operations would naturally occur over repeated iterations in an evolutionary algorithm. However, these operations follow the structure of established edit-distances like the \gls{damerau_levenshtein}. Furthermore, they allow us to efficiently restrict their effects. For instance, we can restrict delete operations to steps that are not padding steps. In contrast, insert operations can be restricted to padding steps only.

Second, we can apply different edit-rates for each edit-type. However, this adds additional complexity and increases the search space for hyperparameters.

Third, using the random sampler automatically disrupts the feasibility for most of the offspring if either of two conditions are met. First, if the log contains categorical/binary event attributes as Gaussian samples cannot reflect these types of random variables. Second, if the vector-space with which event attributes are repsented is too large it becomes less and less likely to sample something within the correct bounds.
For instance, let us consider the example on \autoref{alg:my-evolutionary} again.
However, instead of having 3 event attributes, each event had 100. Then, it becomes extremely difficulty to randomly sample a set that fits the event attribute vectors.

\paragraph{Recombination}
\begin{enumerate}
    \item[FSR:] \emph{Fittest-Survivor-Recombination} strictly determines the survivors among the mutated offsprings and the current population by sorting them in terms of viability.
    The operator guarantees, that the population size remains the same across all iterations.
    Nonetheless, this aproach is subject to getting stuck in local maxima. Mainly, because this recombination scheme does not allow to explore unfavorable solutions that may evolve to better solutions on the long run.
    \item[BBR:] \emph{Best-of-Breed-Recombination} Determines mutants that are better than the average within their generation and adds them to the population. The operator only removes individuals after the maximum population size was reached. Afterwards, the worst individuals are removed to make way for new individuals.
    \item[RR:] \emph{Ranked-Recombination} selects the new population in a different way than the former recombination operators. Instead of using the viability directly, we sort each individuum by every viability component, seperately. This approach allows us to select individuals regardless of the scales of every individual viability measure. We refer to this method as \emph{Ranked-Recombination}. In our order we choose to favor feasibility first. Feasibility values are by far the lowest as they are joint porbability values that become smaller with every multiplication. Second, we favor delta, then sparsity and at last similarity. Mainly, because it is more important to flip the outcome than to change as little as possible; and, it is more important to change as little as event attributes as possible than to become more similar to the factual.
\end{enumerate}

\subsubsection{Naming-Conventions}
We use abbreviations to refer to them in figures, tables, appendices and so on. For instance, \emph{CBI-RWS-OPC-RM-RR} refers to an evolutionary operator configuration, that samples its initial population from the data, probablistically samples parents based on their fitness, crosses them on one point and so on. For the \emph{Uniform-Crossing} operater we additionally indicate its crossing rate using a number. For instance, \emph{CBI-RWS-UC3-RM-RR} is a model using the \emph{Uniform-Crossing} operator. The child receives roughly 30\% of the genom of one parent and 70\% of another parent.

\subsubsection{Hyperparameters}
The evolutionary approach comes with a number of hyperparameters. 

We first discuss the \emph{model configuration}. As shown in this section, there are a \NumEvoCombinations to combine all operators. Depending on each individual operator combination, we might see very different behaviours. For instance, it is obvious, that initiating the population with a random set of values can hardly converge at the same speed as a model which leverages case examples. Similarly, the selection of only the fittest individuals is heavily prone to local optima issues. The decision of the appropriate set of operators is by far the most important in terms of convergence speed and result quality.

The next hyperparameter is the \emph{termination point}. Eventually, most correctly implemented evolutionary algorithms will converge to a local optimum. Especially if only the best individuals are allowed to cross over. If the termination point was chosen to early, then the generated individual will most likely underperform. In contrast, choosing a termination point too far in the future might yield optimal result at the cost of time performance. Furthermore, the existence of local optima may result in very similar solutions in the end. Optimally, we find a termination point, which finds a reasonable middle point.

The \emph{mutation rate} is another hyperparameter. It signifies how much a child can differ from its parent. Again, choosing a rate that is too low does not explore the space as much as it could. In turn, a mutation rate that is too high significantly reduces the chance to converge. The optimal mutation rate allows for exploring novel solutions without immediately pursuing suboptimal solution spaces. Our case is special, as we have four different mutation rates to consider. The change rate, the insertion rate, the deletion rate and the transposition rate. Naturally, these strongly interact. For instance, if the deletion rate is higher than the insertion rate there's a high chance that the sequence will be shorter, if not 0, at the end of its iterative cycles. Mainly, because we remove more events, than we introduce. However, we cannot assume this behaviour across the board as other hyperparamters interplay. Most prominently, the fitness function. Say, we have a high insertion rate but the fitness function rewards shorter sequences. Subsequently, both factors cancel each other out. Hence, the only way to determine the best set of mutation-rates requires an extensive search.

% TODO: Include
% \optional{Another hyperparameter heavily linked to the mutation rate is the ratio between the number generated counterfactuals (\emph{sample-size}) and the population size threshold. If we have an average generation of 10 individuals, and a maximium 1000 in the population, the algorithm is exposed to more mutation opportunities than one which samples 5000 new individuals.} 


\end{document}