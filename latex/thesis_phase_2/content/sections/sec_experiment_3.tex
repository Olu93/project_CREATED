\documentclass[./../../paper.tex]{subfiles}
\graphicspath{{\subfix{./../../figures/}}}

\begin{document}

\section{Determine the best Number of Cycles for the Evolutionary Algorithm }

\subsection{Experimental Setup}
\label{sec:exp3}
Given the combinatory set of configurations from \autoref{sec:exp1} and the chosen hyperparameters of \autoref{sec:exp2}, we explore the effects of different numbers of evolution cycles next. 

We choose to run each configuration for evolution cycles 1, 25, 50, 75 and 100. 
For each configuration we choose the mutation rates and edit rates chosen in \autoref{sec:exp2}. 

We use the resulting viability and feasibility as dependent variables and each count of evolution cycles as independent variable.

The remaining procedure follows the process described in \autoref{sec:exp1}.

\subsection{Results}

\needsfigure{fig:num_cycles_result_1}{This bar-graph shows the effects of increasing the cycle count.}

\needstbl{tbl:viability_cycles}{This table shows the results of the mixed linear model using viability as dependent variable.}

\needstbl{tbl:feasibility_cycles}{This table shows the results of the mixed linear model using feasibility as dependent variable.}

As seen in \autoref{fig:num_cycles_result_1}, \attention{... TBD}.

\subsection{Discussion}
The results can be explained as \attention{... TBD}. Now, we can use this information to select \attention{XX} as optimal number of iteration cycles. 

\end{document}