\documentclass[./../../paper.tex]{subfiles}
\graphicspath{{\subfix{./../../figures/}}}

\begin{document}

We use a function to compute the distance between the factual sequence and the counterfactual candidates. Here, a low distance corresponds to a small change. For reasons explained earlier (\autoref{sec:ssdld}), we want to take the structural distance and the feature distance into account. Henceforth, we will use the previously established \gls{SSDLD}. 
% This distance computes the costs associated with aligning two sequences by taking changes, inserts, deletes and transpositions of elements into account. Each of these alignment operations is typically associated with a cost of 1. This allows us to directly compute the structural difference between two sequences regardless of their lenghts. 
% Instead of computing a cost of 1 for every operation, we compute a cost of a distance between the feature vectors of each step. This cost fuction allows us to take not only the sequential differences into account but also feature differences. 
The similarity distance uses a cost function as specified in \autoref{eq:similarity_measure}.

\begin{align}
    \label{eq:similarity_measure}
    \editCostFunctionBoth      & = L2(a_i, b_j) \\
    a_i,b_j        & \in \mathbb{R}^d \nonumber
\end{align}

\noindent Here, $dist(x,y)$ is an arbitrary distance metric. $i \text{ and } j$ are the indidices of the sequence elements $a \text{ and } b$, respectively.


\end{document}