\documentclass[./../../paper.tex]{subfiles}
\graphicspath{{\subfix{./../../figures/}}}

\begin{document}

\begin{table}
\caption{Shows a factual and the corresponding counterfactual generated.}
\label{tbl:example-cf}
\begin{tabular}{lrlrlrlr}
\toprule
\multicolumn{4}{l}{Factual} & \multicolumn{4}{l}{Counterfactual} \\
Activity & Amount & Resource & Outcome & Activity & Amount & Resource & Outcome \\
\midrule
A-SUBMITTED & 30 000 & 112 & 1 & A-SUBMITTED & 17 353 & 112 & 1 \\
A-PARTLYSUBMITTED & 30 000 & 112 & 1 & A-PARTLYSUBMITTED & 13 680 & 112 & 1 \\
A-DECLINED & 30 000 & 11003 & 1 &  &  &  & 1 \\
\bottomrule
\end{tabular}
\end{table}

\begin{table}
\caption{Shows a factual and the corresponding counterfactual generated. This counterfactual has a non-zero feasibility and has the highest viability among the results generated by the evolutionary algorithm.}
\label{tbl:example-cf-evo-feasibility}
\begin{tabular}{lrlrlrlr}
\toprule
\multicolumn{4}{l}{Factual Sequence} & \multicolumn{4}{l}{Counterfactual Sequence} \\
Activity & Amount & Resource & Outcome & Activity & Amount & Resource & Outcome \\
\midrule
A-SUBMITTED & 20 000 & 112 & 1 &  &  &  &  \\
A-PARTLYSUBMITTED & 20 000 & 112 & 1 &  &  &  &  \\
A-PREACCEPTED & 20 000 & 112 & 1 &  &  &  &  \\
A-ACCEPTED & 20 000 & 9 & 1 &  &  &  &  \\
A-FINALIZED & 20 000 & 9 & 1 &  &  &  &  \\
O-SELECTED & 20 000 & 9 & 1 &  &  &  &  \\
O-CREATED & 20 000 & 9 & 1 &  &  &  &  \\
O-SENT & 20 000 & 9 & 1 &  &  &  &  \\
W-Completeren aanvraag & 20 000 & 9 & 1 & A-SUBMITTED & 18 700 & 112 & 0 \\
W-Nabellen offertes & 20 000 & 1 & 1 & A-PARTLYSUBMITTED & 16 611 & 112 & 0 \\
O-SELECTED & 20 000 & 112 & 1 & A-PREACCEPTED & 27 823 & 112 & 0 \\
O-CANCELLED & 20 000 & 112 & 1 & A-ACCEPTED & 25 283 & 789 & 0 \\
O-CREATED & 20 000 & 112 & 1 & O-SELECTED & 19 682 & 933 & 0 \\
O-SENT & 20 000 & 112 & 1 & O-CREATED & 32 076 & 11319 & 0 \\
W-Nabellen offertes & 20 000 & 112 & 1 & O-SENT & 15 003 & 931 & 0 \\
W-Nabellen offertes & 20 000 & 112 & 1 & W-Completeren aanvraag & 25 120 & 982 & 0 \\
W-Nabellen offertes & 20 000 & 11181 & 1 & W-Nabellen offertes & 33 952 & 112 & 0 \\
W-Nabellen offertes & 20 000 & 113 & 1 & W-Nabellen offertes & 32 465 & 899 & 0 \\
W-Nabellen offertes & 20 000 & 111 & 1 & W-Nabellen offertes & 8 711 & 11181 & 0 \\
W-Nabellen offertes & 20 000 & 982 & 1 & W-Nabellen offertes & 28 585 & other & 0 \\
W-Nabellen offertes & 20 000 & 103 & 1 & W-Nabellen offertes & 16 576 & 11181 & 0 \\
W-Nabellen offertes & 20 000 & 111 & 1 & W-Nabellen offertes & 8 730 & 101 & 0 \\
A-CANCELLED & 20 000 & 111 & 1 &  &  &  &  \\
O-CANCELLED & 20 000 & 111 & 1 & O-CANCELLED & 23 933 & 789 & 0 \\
W-Nabellen offertes & 20 000 & 111 & 1 & W-Nabellen offertes & 40 776 & 982 & 0 \\
 &  &  &  & O-ACCEPTED & 10 998 & 912 & 0 \\
\bottomrule
\end{tabular}
\end{table}

\begin{table}
\caption{Shows a factual and the corresponding counterfactual generated. This counterfactuals was generated by the case-based model. The counterfactual seems far more viable than the one generated by the evolutionary algorithm.}
\label{tbl:example-cf-cbg}
\begin{tabular}{lrlrlrlr}
\toprule
\multicolumn{4}{l}{Factual Sequence} & \multicolumn{4}{l}{Counterfactual Sequence} \\
Activity & Amount & Resource & Outcome & Activity & Amount & Resource & Outcome \\
\midrule
A-SUBMITTED & 15 000 & 112 & 0 &  &  &  &  \\
A-PARTLYSUBMITTED & 15 000 & 112 & 0 &  &  &  &  \\
A-PREACCEPTED & 15 000 & 112 & 0 &  &  &  &  \\
A-ACCEPTED & 15 000 & 9 & 0 &  &  &  &  \\
O-SELECTED & 15 000 & 9 & 0 &  &  &  &  \\
A-FINALIZED & 15 000 & 9 & 0 &  &  &  &  \\
O-CREATED & 15 000 & 9 & 0 & A-SUBMITTED & 10 000 & 112 & 1 \\
O-SENT & 15 000 & 9 & 0 & A-PARTLYSUBMITTED & 10 000 & 112 & 1 \\
W-Completeren aanvraag & 15 000 & 9 & 0 & A-PREACCEPTED & 10 000 & 112 & 1 \\
W-Nabellen offertes & 15 000 & 103 & 0 & W-Completeren aanvraag & 10 000 & 111 & 1 \\
W-Nabellen offertes & 15 000 & 11181 & 0 & A-ACCEPTED & 10 000 & 111 & 1 \\
W-Nabellen offertes & 15 000 & 982 & 0 & O-SELECTED & 10 000 & 111 & 1 \\
O-SENT-BACK & 15 000 & 11259 & 0 & A-FINALIZED & 10 000 & 111 & 1 \\
W-Nabellen offertes & 15 000 & 11259 & 0 & O-CREATED & 10 000 & 111 & 1 \\
W-Valideren aanvraag & 15 000 & 138 & 0 & O-SENT & 10 000 & 111 & 1 \\
O-ACCEPTED & 15 000 & 138 & 0 & W-Completeren aanvraag & 10 000 & 111 & 1 \\
A-REGISTERED & 15 000 & 138 & 0 & W-Nabellen offertes & 10 000 & 111 & 1 \\
A-APPROVED & 15 000 & 138 & 0 & O-CANCELLED & 10 000 & 111 & 1 \\
A-ACTIVATED & 15 000 & 138 & 0 & A-CANCELLED & 10 000 & 111 & 1 \\
W-Valideren aanvraag & 15 000 & 138 & 0 & W-Nabellen offertes & 10 000 & 111 & 1 \\
\bottomrule
\end{tabular}
\end{table}

\begin{table}
\caption{Shows a factual and the corresponding counterfactual generated. }
\label{tbl:example-cf-evo-rr}
\begin{tabular}{lrlrlrlr}
\toprule
\multicolumn{4}{l}{Factual Sequence} & \multicolumn{4}{l}{Counterfactual Sequence} \\
Activity & Amount & Resource & Outcome & Activity & Amount & Resource & Outcome \\
\midrule
A-SUBMITTED & 8 500 & 112 & 0 &  &  &  &  \\
A-PARTLYSUBMITTED & 8 500 & 112 & 0 &  &  &  &  \\
A-PREACCEPTED & 8 500 & 112 & 0 &  &  &  &  \\
W-Completeren aanvraag & 8 500 & 119 & 0 &  &  &  &  \\
A-ACCEPTED & 8 500 & 119 & 0 &  &  &  &  \\
O-SELECTED & 8 500 & 119 & 0 &  &  &  &  \\
A-FINALIZED & 8 500 & 119 & 0 &  &  &  &  \\
O-CREATED & 8 500 & 119 & 0 &  &  &  &  \\
O-SENT & 8 500 & 119 & 0 &  &  &  &  \\
W-Completeren aanvraag & 8 500 & 119 & 0 &  &  &  &  \\
O-SENT-BACK & 8 500 & 11259 & 0 &  &  &  &  \\
W-Nabellen offertes & 8 500 & 11259 & 0 &  &  &  &  \\
O-ACCEPTED & 8 500 & 629 & 0 & A-SUBMITTED & 9 188 & 112 & 1 \\
A-REGISTERED & 8 500 & 629 & 0 & A-PARTLYSUBMITTED & 3 040 & 112 & 1 \\
A-APPROVED & 8 500 & 629 & 0 & A-PREACCEPTED & 15 627 & 112 & 1 \\
A-ACTIVATED & 8 500 & 629 & 0 & W-Completeren aanvraag & 20 205 & 179 & 1 \\
W-Valideren aanvraag & 8 500 & 629 & 0 & W-Nabellen offertes & 8 097 & 138 & 1 \\
\bottomrule
\end{tabular}
\end{table}


Here, in this concrete example, we can see the bias of the viability measure. The best counterfactual is the shortest possible. We also see, that the counterfactual amount is smaller than the factual amount. The counterfactual resource is the same as the factual. Furthermore, the model now predicts the opposite outcome, of what it would have predicted beforehand.



\end{document}