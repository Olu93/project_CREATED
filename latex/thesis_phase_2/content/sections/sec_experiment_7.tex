\documentclass[./../../paper.tex]{subfiles}
\graphicspath{{\subfix{./../../figures/}}}

\begin{document}

In the result tables you can see 4 factuals with the best counterfactual the model produced. 

\begin{table}
\caption{Shows a factual and the corresponding counterfactual generated. This counterfactual was generated by the evolutionary algorithm. It is the result which appears to have the highest viability score.}
\label{tbl:example-cf-evo}
\begin{tabular}{lrlrlrlr}
\toprule
\multicolumn{4}{l}{Factual Sequence} & \multicolumn{4}{l}{Counterfactual Sequence} \\
Activity & Amount & Resource & Outcome & Activity & Amount & Resource & Outcome \\
\midrule
A-SUBMITTED & 12 500 & 112 & 1 &  &  &  &  \\
A-PARTLYSUBMITTED & 12 500 & 112 & 1 &  &  &  &  \\
A-PREACCEPTED & 12 500 & 112 & 1 &  &  &  &  \\
W-Completeren aanvraag & 12 500 & 11189 & 1 &  &  &  &  \\
W-Completeren aanvraag & 12 500 & 11189 & 1 & A-SUBMITTED & 23 292 & 112 & 0 \\
W-Completeren aanvraag & 12 500 & 11121 & 1 & A-PARTLYSUBMITTED & 15 264 & 112 & 0 \\
W-Completeren aanvraag & 12 500 & 11121 & 1 & A-PREACCEPTED & 14 758 & 112 & 0 \\
A-ACCEPTED & 12 500 & 11169 & 1 &  &  &  &  \\
A-FINALIZED & 12 500 & 11169 & 1 & W-Completeren aanvraag & 12 282 & 971 & 0 \\
O-SELECTED & 12 500 & 11169 & 1 & A-FINALIZED & 20 306 & 110 & 0 \\
O-CREATED & 12 500 & 11169 & 1 & O-SELECTED & 4 359 & 11122 & 0 \\
O-SENT & 12 500 & 11169 & 1 & O-CREATED & 2 624 & 111 & 0 \\
W-Completeren aanvraag & 12 500 & 11169 & 1 & O-SENT & 3 188 & 863 & 0 \\
O-SENT-BACK & 12 500 & 129 & 1 & W-Completeren aanvraag & 12 425 & 912 & 0 \\
W-Nabellen offertes & 12 500 & 129 & 1 & O-SENT-BACK & 24 866 & 939 & 0 \\
O-DECLINED & 12 500 & 629 & 1 & W-Nabellen offertes & 12 028 & 862 & 0 \\
A-DECLINED & 12 500 & 629 & 1 & W-Valideren aanvraag & 23 659 & 11111 & 0 \\
W-Valideren aanvraag & 12 500 & 629 & 1 &  &  &  &  \\
 &  &  &  & A-ACTIVATED & 1 691 & 138 & 0 \\
\bottomrule
\end{tabular}
\end{table}


\begin{table}
\caption{Shows a factual and the corresponding counterfactual generated. This counterfactual has a non-zero feasibility and has the highest viability among the results generated by the evolutionary algorithm.}
\label{tbl:example-cf-evo-feasibility}
\begin{tabular}{lrlrlrlr}
\toprule
\multicolumn{4}{l}{Factual Sequence} & \multicolumn{4}{l}{Counterfactual Sequence} \\
Activity & Amount & Resource & Outcome & Activity & Amount & Resource & Outcome \\
\midrule
A-SUBMITTED & 20 000 & 112 & 1 &  &  &  &  \\
A-PARTLYSUBMITTED & 20 000 & 112 & 1 &  &  &  &  \\
A-PREACCEPTED & 20 000 & 112 & 1 &  &  &  &  \\
A-ACCEPTED & 20 000 & 9 & 1 &  &  &  &  \\
A-FINALIZED & 20 000 & 9 & 1 &  &  &  &  \\
O-SELECTED & 20 000 & 9 & 1 &  &  &  &  \\
O-CREATED & 20 000 & 9 & 1 & A-SUBMITTED & 23 930 & 112 & 0 \\
O-SENT & 20 000 & 9 & 1 & A-PARTLYSUBMITTED & 33 067 & 112 & 0 \\
W-Completeren aanvraag & 20 000 & 9 & 1 & A-PREACCEPTED & -6 864 & 110 & 0 \\
W-Nabellen offertes & 20 000 & 1 & 1 & O-SELECTED & 21 520 & 982 & 0 \\
O-SELECTED & 20 000 & 112 & 1 &  &  &  &  \\
O-CANCELLED & 20 000 & 112 & 1 &  &  &  &  \\
O-CREATED & 20 000 & 112 & 1 & W-Nabellen offertes & 29 499 & 913 & 0 \\
O-SENT & 20 000 & 112 & 1 &  &  &  &  \\
W-Nabellen offertes & 20 000 & 112 & 1 & O-SELECTED & 37 705 & 913 & 0 \\
W-Nabellen offertes & 20 000 & 112 & 1 &  &  &  &  \\
W-Nabellen offertes & 20 000 & 11181 & 1 & O-SENT & -684 & nan & 0 \\
W-Nabellen offertes & 20 000 & 113 & 1 & W-Nabellen offertes & 20 730 & 881 & 0 \\
W-Nabellen offertes & 20 000 & 111 & 1 & W-Nabellen offertes & -2 233 & 11119 & 0 \\
W-Nabellen offertes & 20 000 & 982 & 1 & W-Nabellen offertes & -539 & 11189 & 0 \\
W-Nabellen offertes & 20 000 & 103 & 1 & W-Nabellen offertes & 19 690 & 982 & 0 \\
W-Nabellen offertes & 20 000 & 111 & 1 & W-Nabellen offertes & 10 543 & 862 & 0 \\
A-CANCELLED & 20 000 & 111 & 1 & W-Nabellen offertes & 14 978 & 935 & 0 \\
O-CANCELLED & 20 000 & 111 & 1 & W-Nabellen offertes & 24 171 & 111 & 0 \\
W-Nabellen offertes & 20 000 & 111 & 1 & A-CANCELLED & 33 138 & 112 & 0 \\
 &  &  &  & A-REGISTERED & 5 386 & 119 & 0 \\
\bottomrule
\end{tabular}
\end{table}


\begin{table}
\caption{Shows a factual and the corresponding counterfactual generated.}
\label{tbl:example-cf}
\begin{tabular}{lrlrlrlr}
\toprule
\multicolumn{4}{l}{Factual} & \multicolumn{4}{l}{Counterfactual} \\
Activity & Amount & Resource & Outcome & Activity & Amount & Resource & Outcome \\
\midrule
A-SUBMITTED & 10 000 & 112 & 0 & A-SUBMITTED & 14 220 & 112 & 1 \\
A-PARTLYSUBMITTED & 10 000 & 112 & 0 & A-PARTLYSUBMITTED & 14 210 & 112 & 1 \\
A-PREACCEPTED & 10 000 & 112 & 0 &  &  &  & 1 \\
A-ACCEPTED & 10 000 & 10809 & 0 &  &  &  & 1 \\
O-SELECTED & 10 000 & 10809 & 0 &  &  &  & 1 \\
A-FINALIZED & 10 000 & 10809 & 0 &  &  &  & 1 \\
O-CREATED & 10 000 & 10809 & 0 &  &  &  & 1 \\
O-SENT & 10 000 & 10809 & 0 &  &  &  & 1 \\
W-Completeren aanvraag & 10 000 & 10809 & 0 &  &  &  & 1 \\
O-SENT-BACK & 10 000 & 10809 & 0 &  &  &  & 1 \\
W-Nabellen offertes & 10 000 & 10809 & 0 &  &  &  & 1 \\
O-ACCEPTED & 10 000 & 10809 & 0 &  &  &  & 1 \\
A-APPROVED & 10 000 & 10809 & 0 &  &  &  & 1 \\
A-ACTIVATED & 10 000 & 10809 & 0 &  &  &  & 1 \\
A-REGISTERED & 10 000 & 10809 & 0 &  &  &  & 1 \\
\bottomrule
\end{tabular}
\end{table}


% \begin{table}
\caption{Shows a factual and the corresponding counterfactual generated. }
\label{tbl:example-cf-evo-rr}
\begin{tabular}{lrlrlrlr}
\toprule
\multicolumn{4}{l}{Factual Sequence} & \multicolumn{4}{l}{Counterfactual Sequence} \\
Activity & Amount & Resource & Outcome & Activity & Amount & Resource & Outcome \\
\midrule
A-SUBMITTED & 8 500 & 112 & 0 &  &  &  &  \\
A-PARTLYSUBMITTED & 8 500 & 112 & 0 &  &  &  &  \\
A-PREACCEPTED & 8 500 & 112 & 0 &  &  &  &  \\
W-Completeren aanvraag & 8 500 & 119 & 0 &  &  &  &  \\
A-ACCEPTED & 8 500 & 119 & 0 &  &  &  &  \\
O-SELECTED & 8 500 & 119 & 0 &  &  &  &  \\
A-FINALIZED & 8 500 & 119 & 0 &  &  &  &  \\
O-CREATED & 8 500 & 119 & 0 &  &  &  &  \\
O-SENT & 8 500 & 119 & 0 &  &  &  &  \\
W-Completeren aanvraag & 8 500 & 119 & 0 &  &  &  &  \\
O-SENT-BACK & 8 500 & 11259 & 0 &  &  &  &  \\
W-Nabellen offertes & 8 500 & 11259 & 0 &  &  &  &  \\
O-ACCEPTED & 8 500 & 629 & 0 & A-SUBMITTED & 9 188 & 112 & 1 \\
A-REGISTERED & 8 500 & 629 & 0 & A-PARTLYSUBMITTED & 3 040 & 112 & 1 \\
A-APPROVED & 8 500 & 629 & 0 & A-PREACCEPTED & 15 627 & 112 & 1 \\
A-ACTIVATED & 8 500 & 629 & 0 & W-Completeren aanvraag & 20 205 & 179 & 1 \\
W-Valideren aanvraag & 8 500 & 629 & 0 & W-Nabellen offertes & 8 097 & 138 & 1 \\
\bottomrule
\end{tabular}
\end{table}


Across all examples, we see the bias of the viability measure. Every counterfactual is shorter in length than their factual counterpart. 
We also see, that the Amount fluctuates heavily for the evolutionary generator 
Similar, holds for the resource field. All models manage to capture the first three activities and its resource. Also, the counterfactual outcome is the opposite of the factual outcome in all cases. Hence, the each generator successfully inverts the model prediction. Furthermore, each model captures the starting events, quite well.

We see in \autoref{tbl:example-cf-evo} and \autoref{tbl:example-cf-evo-feasibility}. 
Furthermore, the model with the highest feasibility among the \ModelEVO generated counterfactuals, is also the one with the highest viability. 

\autoref{tbl:example-cf-cbg} displays a counterfactual generated by the \ModelCBG. We see that its result appears to be more viable upon inspection. Here, all Amount variables are below the factual Amount. \citeauthor{hsieh_DiCE4ELInterpretingProcess_2021} interprets this as a reasonable thought. From it, we can interpret, that we have better chances of a successful loan application process if we request a lower loan. This result tells us, that the viability measure \emph{does} capture a notion of viability. However, it is not enough to generated realistic counterfactuals for models that optimize it. 



% The results also does not seem to be realistic as an O-SELECTED cannot follow after an A-ACCEPTED activity. The counterfactual managed to be the best, despite having zero feasibility. 








\end{document}