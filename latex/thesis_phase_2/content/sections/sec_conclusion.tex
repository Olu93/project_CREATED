\documentclass[./../../paper.tex]{subfiles}
\graphicspath{{\subfix{./../../figures/}}}

\begin{document}
In conclusion, we researched \textbf{how we generate counterfactual sequences while incorporating structural differences between the factual sequence and the counterfactual sequence}. We showed it is possible to use a viability measure and incorporate structural differences. We can also use an evolutionary algorithm to optimise this viability measure. 

Concerning RQ1\footnotemark[1], we can design a \Gls{SSDLD} that can compute distances even if the semi-structured data is multivariate. We employ an evolutionary model to achieve this goal in Experiment 1 (\autoref{sec:experiment1}).

For RQ2\footnotemark[2], we see the extent to which our counterfactuals fulfill viability. 
We show that by conducting Experiment 2 (\autoref{sec:experiment2}), which confirms the hypothesis that our models outperform both random-based (RQ1-H1) and case-based approaches (RQ1-H2).  

For RQ3\footnotemark[3], we showed that our counterfactuals are viable. 
We confirm that in Experiment 3 (\autoref{sec:experiment3}) and Experiment 4 (\autoref{sec:experiment4}).
These experiments show that our counterfactuals are not only comparable to existing work in the literature (RQ3-H1) but can even align with factuals to make both more comparable.

To summarize, we answered all research questions and confirmed all hypotheses. Domain experts can still contest the viability of the counterfactuals. However, we believe that counterfactuals primarily explain the model we attempt to understand. Therefore, they are a valid and transparent reflection of a particular model. Furthermore, we show it is worth pursuing more research and insights into the counterfactual generation of processes. Examples within this thesis showed that processes are a ubiquitous part of our life. Many things can be understood as a process. Hence, shying away from complicated problems like multivariate sequence problems heavily limits our progress and understanding of the cause and effect relations within our daily lives. 

\footnotetext[1]{How can we employ existing methods to compute viability so that its optimization incorporates information about the structure of the sequence?}
\footnotetext[2]{To what extent can we generate counterfactuals that fulfill the criteria to be viable?}
\footnotetext[3]{How does an algorithm which optimizes multiple viability quality metrics perform against other approaches?}

\end{document}
