\documentclass[./../../paper.tex]{subfiles}
\graphicspath{{\subfix{./../../figures/}}}

\begin{document}

Some of the results are not surprising. We expected the feasibility to favor smaller sequences. This characteristic can be an advantag for use cases, such as medicine. The changing loan amount was expected, as well. We did not implement any safeguard to prevent this. Throughout a process sequence, the amount acts more similar to a constant or categorical variable than a numerical one if we look at the factual examples. 

We are also not surprised, all models manage to capture the first few activities. These are exactly the same accross all cases. Hence, there no variation. A model not recognizing this characteristic of the proce cannot be deemed viable.

All models successfully manage to flip the out come of the prediction model, despite seeming unviable at first glance. This observation tells us that these counterfactuals tell us more about the model rather than the true process. 

The \ModelCBG does seem to generate viable counterfactuals. Hence, its results are not only close to the original, but also reasonable.   

All in all, we cannot claim that the generator model tells us anything about the true process. The results are far too inconsistent to see any pattern. 

\end{document}