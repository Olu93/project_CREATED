\documentclass[./../../paper.tex]{subfiles}
\graphicspath{{\subfix{./../../figures/}}}

\begin{document}


Given the current viability function we can already determine the optimal counterfactual:
\begin{displayquote}
    The optimal counterfactual flips the strongly expected factual outcome of a model to a desired outcome, maintaining the same trajectory as the factual in terms of events, with minimal changes its event attributes, while remaining feasible according to the data.
\end{displayquote}

\noindent The elements that fulfill these criteria make up the pareto surface of this multi-valued viability function. If each of the values are scaled a range between 0 and 1, the theoretical ceiling is 4. This value is only possible if we can flip the outcome of a factual sequence without chaning it. As this is naturally impossible for deterministic model predictions, the viability has to be lower than 4. 

Furthermore, we can already postulate, that a viability of 2 is an important threshold. If we score the viability of a factual against itself, a normalised sparcity and similarity value have to at its maximal value of 1. In contrast, the improvement has to be 0. The feasibility is 0 depending on whether the factual was used to estimate the data distribution or not. \optional{With these observations in mind, we determine that any counterfactual with a viability of at least to is a considerable counterfactual.} 

\end{document}