\documentclass[./../../paper.tex]{subfiles}

\graphicspath{{\subfix{./../../figures/}}}



\begin{document}



The results for \autoref{fig:exp5-winner} show that both evolutionary algorithms outperform the competition across all datasets and against all baselines. The fact that sparsity and similarity are the main drivers for this consistent improvement indicates a higher structural alignment between counterfactual and factual. 



This remarkable result shows that the algorithm can outperform baselines regardless of the process log and its length.



The underperformance of the random model was expected. In \autoref{sec:viability}, we indicated that viable algorithms must at least reach a viability of 2. 

Furthermore, we expected the search space for the \ModelRNG is too vast to find viable results. 

The fact that 8 of 9 datasets showed that the random model cannot exceed the threshold of two supports this claim. Additional support is the observation that every \ModelCBG reaches at least 2.



% The results hint at the empirical upper and lower bounds of the viability measure. None of the median viabilities exceeds 3. We discussed the reasons in \autoref{sec:viability}. Mainly, the you  

\end{document}

7
All suggestions
