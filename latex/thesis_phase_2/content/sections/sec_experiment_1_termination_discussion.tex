\documentclass[./../../paper.tex]{subfiles}
\graphicspath{{\subfix{./../../figures/}}}

\begin{document}


The results are not surprising. The longer the algorithm runs the closer it gets to a local minimum. 
We expect every evolutionary algorithm to converge at some point, as only the best within the population are chosen for the next iteration. If the model does not include enough non-deterministic components the results collapse to one optimal case in terms of structure. Hence, the counterfactual activities remain unchanged for the rest of the generation process. 
The events ratio should optimally approach a number around 0.5 if the factuals are evenly distributed in length. All model-configurations seemingly follow this trajectory. 
However, models (\optional{RI-TS-TPC-SBM-RR}) falls below this level. This coincides with its sharp rise in feasibility. We assume this behavior relates to a bias of the feasibility measure towards shorter sequences.
The rise and decline of \optional{CBI-RWS-OPC-SBM-BBR} shortly before overtaking all other models in terms of similarity and sparsity indicate a trade-off between how close the counterfactual is to the factual and how feasible it is.     


% In terms of similarity and sparsity all models behave similar. This is no surprise as both measures are ineherently interlinked. 
% We see that the randomly initiated models (RI-x) decrease the amount of events they generate. 
% Case-based initiated models appear to slightly gain more events. Although,  \attention{CBI-RWS-OPC-SBN-BBR} appears that reaches its saturation point significantly later (\attention{100}th cycle).
% Interestingly, the \attention{CBI-RWS-OPC-SBM-BBR} model struggles to maintain feasibility and collapses to near 0 after the \attention{100th} iterative cycle. 
% Another surprise is the steep ascention of only model that uses tournament selection (RI-TS-TPC-SBM-RR) towards the end of the generation process. The  model even overtakes the model that leads the model-configurations in terms of \attention{viability}. 
% Furthermore, we see that {CBI-ES-UC-SBM-RR} has the highest feasibility among all models. However, it also quickly converges after 50 iterative cycles.    

% Most of the result were expected. The longer the algorithm runs, the more it narrows its solution space. Typically, this is a favorable characteristic. However, for counterfactual explanations, this advatange becomes a nuisance. We generally seek more diverse solutions. As discussed in \autoref{sec:viability}, we favor diverse solutions, but this aspect is not reflected in the viability measure. 

% The low dispersion cases are probably solutions that are stuck in local optima. As mentioned earlier, this is a common flaw for evolutionary algorithms.

% With regards to the low viability outliers, we can safely ignore those as we are mostly interested in the higher ranking counterfactuals.

% TODO: Need to align the distinction between viability measure and viability components. 
% TODO: Need to align naming of delta towards somethin better. Maybe Classification measure.
% TODO: Need to change viability and componets to title case with emphasis.

For the next experiments we are going to use \attention{50} as a termination point. It appears to be a reasonable point in which most models reach their highest viability yield and have not converged yet. We do not seek convergence, as it we want to maintain the diversity of our counterfactuals. 
\end{document}