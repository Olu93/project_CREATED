\documentclass[./../../paper.tex]{subfiles}
\graphicspath{{\subfix{./../../figures/}}}

\begin{document}

This baseline resembles the random baseline. However, we use the feasibility model to guide the random search for the generation of counterfactuals. We refer to the model specified in \autoref{eq:feasibility_measure}. The sampling procedure utilises the model structure for the sampling process. We first generate a random seed of possible starting events ($\prob{e_0}$). Afterwards, we randomly sample subsequent events by iteratively sampling new activities according to the transition probabilities we gathered from the data ($\prod_1^T \cprob{e_t}{e_{t-1}}$). Given the sequence, we simply sample the features per event from $\cprob{f_t}{e_t}$. 

% For modelling the probabilities, we use a mixture of probability distributions. For binary encoded variables, we use Bernoulli distributions. For numerical event attributes, a multivariate gaussian. We assume that each distribution is independent. There, we only have to compute their product.

\end{document}